% Based on template: http://www.maths.lth.se/matematiklth/exjobb/exjobbarresurs/index.html

\documentclass[a4paper]{article}   % TODO: Use IEEEtran instead?

\usepackage[smartEllipses]{markdown}  % For markdown
\usepackage{comment}

% Encoding and languages
\usepackage[T1]{fontenc}        % För svenska bokstäver
%\usepackage[swedish]{babel}    %Svenska skrivregler och rubriker

% Graphics
\usepackage{epsfig}
%\usepackage[dvips]{graphics}

% References
\usepackage[backend=biber,style=numeric]{biblatex}
\addbibresource{sample.bib}
\addbibresource{zotero.bib}

\title{MSc Thesis proposal --- A study of the relationship between device and brain activity}
\author{Erik Bjäreholt \\(dat13ebj@student.lu.se, erik@bjareho.lt)}
\date{\today}

\begin{document}
\maketitle

\noindent Student: Erik Bjäreholt (dat13ebj@student.lu.se)

\noindent Thesis supervisor: Markus Borg

\noindent Thesis examiner: Elizabeth Bjarnason

\noindent Start date: 10 June (realistic?)

\noindent End date: 30 September (realistic?)

\pagebreak

\tableofcontents

\pagebreak

\begin{comment}
\section{Requirements for this document}

The process for CS students: http://cs.lth.se/examensarbete/hur-gaar-det-till/
General CS dep resource: http://cs.lth.se/examensarbete/
General LTH resource: http://www.student.lth.se/studieinformation/examensarbete/examensarbetsprocessen/

 - [ ] Arbetstitel, inblandades namn och kontaktuppgifter samt preliminärt start- och slutdatum.
 - [ ] Bakgrund/kontext och motiv för examensarbetet.
 - [ ] Övergripande mål och problemställningar/forskningsfrågor.
 - [ ] Angreppssätt/metodik och metoder.
 - [ ] Vetenskaplig grund och beprövad erfarenhet som examensarbetet ska bygga vidare på. Detta kan t ex beskrivas i form av ett par nyckelreferenser till artiklar eller annat underlag.
 - [ ] Hur förväntas examensarbetet bidra till kunskapsutvecklingen?
 - [ ] Preliminär beskrivning av resurser som krävs för arbetets genomförande, t ex arbetsplats och utrustning, och hur dessa ordnas och finns tillgängliga.

\end{comment}

\section{Background}

\begin{markdown}
People spend more time than ever using computing devices (TODO: reference). As services, entertainment, and work, moves online this trend is expected to continue. Studies

Data on how people spend their screen time, and how that varies with demographics, is not publicly available (TODO: reference).

Furthermore, how different computer activities affects the user behaviorally and neurologically is of interest for many areas of research, including:

 - the impact of "screen time" for adolescents (TODO: reference)
 - attention span among media multitasking adults (TODO: reference)
 - depression

There are companies (RescueTime, etc.) who offer automated time tracking as a service. These services generally function by having the user install a program on their device which tracks the active application and sends the data to their servers for storage and analysis. The user can then view their data in a dashboard on the service providers website. There services are marketed towards teams and professionals, who generally want to keep track of individual and team productivity.

However, by collecting detailed and non-anonymized behavioral data on the user these services bring significant privacy concerns, especially in cases where the data is shared with a team or an employer.

With the advancement of Brain-Computer Interfaces, the relationship between device and brain activity is becoming even more tightly connected.

Functional brain imaging methods such as EEG, fNIRS, fMRI...

As a starting point for the thesis, the
\end{markdown}

\section{Problem description, research goals and questions}
\begin{markdown}

 - What about measuring flow?
 - What about measuring attention/distractibility?

\end{markdown}

\section{Methodology}

\section{Scientific contributions}
\begin{markdown}

 - The open source automated time-tracker ActivityWatch.
 - Relationships between device activity and brain activity, as measured by EEG.

\end{markdown}

\section{Resources}
\begin{markdown}

 - Data collected with ActivityWatch

\end{markdown}


\section{References}

\printbibliography{}

\end{document}
