% Uncomment to use official LTH thesis template
\def\UseCSLTH{true}

\PassOptionsToPackage{table}{xcolor}

\ifx\UseCSLTH\undefined
    \documentclass[a4paper]{article}
\else
    \documentclass[nofilelist]{cslthse-msc}
\fi

% Based on templates:
%  - http://www.maths.lth.se/matematiklth/exjobb/exjobbarresurs/index.html
%  - https://bitbucket.org/flavius_gruian/msccls/

\usepackage[smartEllipses]{markdown}  % For markdown
\def\markdownOptionOutputDir{build}  % Needed, see https://github.com/Witiko/markdown/issues/6#issuecomment-328699108

\usepackage{array}
\usepackage{hyperref}
\usepackage{makecell}
\usepackage{comment}
\usepackage{xargs}
\usepackage{verbatim}
\usepackage{subfig}
\usepackage{booktabs}
\usepackage{changepage}
\usepackage{rotating, graphicx}
\usepackage{multirow}
\usepackage{pdflscape}
\usepackage{afterpage}
\usepackage[export]{adjustbox}
\usepackage[dvipsnames]{xcolor}
\usepackage[bottom]{footmisc}
\usepackage[colorinlistoftodos, prependcaption, textsize=tiny]{todonotes}

% References
\usepackage[backend=biber, style=numeric, sorting=none]{biblatex}
\usepackage{cleveref}
% Turn off split bibliography warning
\BiblatexSplitbibDefernumbersWarningOff

% ignore badness warnings (not working?)
%\vbadness=10000
%\hbadness=10000

% Add subsubsubsection
% As per https://tex.stackexchange.com/a/60212/36302
\usepackage{titlesec}
\setcounter{secnumdepth}{4}
\titleformat{\paragraph}
{\normalfont\normalsize\bfseries}{\theparagraph}{1em}{}
\titlespacing*{\paragraph}
{0pt}{3.25ex plus 1ex minus .2ex}{1.5ex plus .2ex}

% Better formating for values and units
% From https://tex.stackexchange.com/questions/79141/is-there-a-designated-symbol-for-the-negative-sign-in-say-16
\DeclareUnicodeCharacter{2212}{\textminus}% requires a unicode capable editor
\usepackage{siunitx}
\sisetup{
   detect-mode,
   detect-family,
   detect-inline-family=math,
}

% Ensures floats stay in their section
\usepackage[section]{placeins}

% Code snippets
\usepackage[outputdir=build]{minted}
\usemintedstyle{vs}

\usepackage[english]{datetime2}
\DTMnewdatestyle{dashdate}{%
  \renewcommand{\DTMdisplaydate}[4]{\number##1-\DTMenglishshortmonthname{##2}-\number##3}%
  \renewcommand{\DTMDisplaydate}{\DTMdisplaydate}%
}
\DTMsetdatestyle{iso}

% From: https://tex.stackexchange.com/a/178806/36302
\newcommandx{\add}[2][1=]{\todo[linecolor=red, backgroundcolor=red!25, bordercolor=red, inline, #1]{\textbf{Add:} #2}}
\newcommandx{\unsure}[2][1=]{\todo[linecolor=red, backgroundcolor=red!25, bordercolor=red, #1]{\textbf{Unsure:} #2}}
\newcommandx{\change}[2][1=]{\todo[linecolor=blue, backgroundcolor=blue!25, bordercolor=blue, #1]{\textbf{Change:} #2}}
\newcommandx{\info}[2][1=]{\todo[linecolor=OliveGreen, backgroundcolor=OliveGreen!25, bordercolor=OliveGreen, #1]{\textbf{Info:} #2}}
\newcommandx{\improvement}[2][1=]{\todo[linecolor=Plum, backgroundcolor=Plum!25,bordercolor=Plum, #1]{\textbf{Improve:} #2}}
\newcommandx{\thiswillnotshow}[2][1=]{\todo[disable, #1]{#2}}

% For adding source captions to figures.
% From: https://tex.stackexchange.com/a/246285/36302
\newcommand{\source}[1]{\vspace{-0.3cm} \caption*{\footnotesize{Source: \textit{{#1}}}}}

% Formatting
\setlength{\parindent}{0pt}
\setlength{\parskip}{1em}

% Encoding and languages
\usepackage[utf8]{inputenc}
\usepackage[english]{babel}
\usepackage[T1]{fontenc}        % För svenska bokstäver
%\usepackage[swedish]{babel}    %Svenska skrivregler och rubriker

% Graphics
\usepackage{epsfig}
%\usepackage[dvips]{graphics}

\newcommandx{\orcid}[1]{\href{https://orcid.org/#1}{\includegraphics[width=0.7em]{img/orcid-icon.png}}}

\newcommand\myshade{85}
\colorlet{mylinkcolor}{violet}
\colorlet{mycitecolor}{YellowOrange}
\colorlet{myurlcolor}{Aquamarine}

\hypersetup{%
  linkcolor  = black, %mylinkcolor!\myshade!black,
  citecolor  = mycitecolor!\myshade!black,
  urlcolor = myurlcolor!\myshade!black,
  colorlinks = true,
}

% wide page for side by side figures, tables, etc
\newlength{\offsetpage}
\setlength{\offsetpage}{1.0cm}
\newenvironment{widepage}{\begin{adjustwidth}{-\offsetpage}{-\offsetpage}%
    \addtolength{\textwidth}{2\offsetpage}}%
{\end{adjustwidth}}

% References
\bibliography{zotero}
\bibliography{misc}
\DeclareBibliographyCategory{cited}
\AtEveryCitekey{\addtocategory{cited}{\thefield{entrykey}}}

\defbibheading{notcited}{\section*{Further Reading}}

\def\mytitleen{Classifying brain activity using electroencephalography and automated tracking of computer use}
\def\mytitlesv{Klassificering av hjärnaktivitet med elektroencephalografi och automatiserad loggning av datoranvändning}

\ifx\UseCSLTH\undefined
    \title{%
        \small DRAFT \today \\
        \small The latest version is available at \href{https://erik.bjareholt.com/thesis/thesis.pdf}{erik.bjareholt.com/thesis/thesis.pdf}\\
        \large --- \\
        \large \par M.Sc. Thesis\\
        \huge \mytitleen\\
    }
    \author{Erik Bjäreholt \orcid{0000-0003-1350-9677} \\(erik@bjareho.lt, dat13ebj@student.lu.se)}
    \date{\today}
\else
    \title{\mytitleen}
    \svensktitel{\mytitlesv}
    \subtitle{Subtitle in progress}

    \date{\today}
    %\date{January 16, 2015}

    \student{Erik Bjäreholt}{erik@bjareho.lt}

    \company{RISE}
    \supervisor{Markus Borg, \texttt{markus.borg@\{\href{mailto:markus.borg@cs.lth.se}{cs.lth.se}, \href{mailto:markus.borg@ri.se}{ri.se}\}}}
    \examiner{Elizabeth Bjarnarson, \href{mailto:elizabeth.bjarnason@cs.lth.se}{\texttt{elizabeth.bjarnason@cs.lth.se}}}

    %\geometry{showframe}

    \thesisnumber{LU-CS-EX: 2023-79} % Magic Number! Do not change unless Birger Swahn asks you to do so!
    % default is Master. Uncomment the following for "kandidatarbete"/Bachelor's thesis
    %\thesistype{Bachelor}{Kandidatarbete}

    %\onelinetitle
    %\twolinestitle
    \threelinestitle
    %\fourlinestitle

    \theabstract{We investigate the ability of EEG to distinguish between different activities users engage in on their devices, building on previous research which showed a considerable difference in brain activity between code- and prose-comprehension, as well as differences during code- and prose-synthesis. We perform a replication study and improve upon past results using state-of-the-art machine learning classifiers based on Riemannian geometry.

Furthermore, we extend the scope of previous work by introducing the automated time tracking application ActivityWatch, to track the device activities that the user is engaging in. This lets us label EEG data with naturalistic device activity, which we then use to train classifiers to discern activities such as code writing vs prose writing, or work vs media consumption. Our results indicate that a consumer-grade EEG device can discern between different activities that a user performs at the computer. Among other results, we show that not only can code and prose \emph{comprehension} be distinguished, but also code and prose \emph{writing}.

% Our results show... 
% Our results can be used by software engineers and knowledge workers seeking to...
% Our results pave the way to better understand the minds of computer users in general, and software engineers, in particular, especially at work.
% Our results indicate that a consumer-grade EEG device can provide actionable insights into what a user is doing at the computer. More specifically, we show that\ldots"

% Our results confirm the findings of the original study, i.e., EEG data can be used to distinguish between code and prose-comprehension. Furthermore, the classifier based on Riemannian geometry outperforms the bandpass-features used by Fucci et al.~in terms of classification accuracy.

% Our results indicate that a consumer-grade EEG device can provide actionable insights into what a user is doing at the computer. More specifically, we show that we can train a classifier to discern between several types of device activity.

% In our naturalistic experiments, our results indicate that a consumer-grade EEG device can discern between different activities that a user performs at the computer. Among other results, we show that not only can code and prose \emph{comprehension} be distinguished, but also code and prose \emph{writing}.

A full replication package, including source code and a sample dataset, is available at \href{https://github.com/ErikBjare/thesis}{github.com/ErikBjare/thesis}
}

    \acknowledgements{\begin{itemize}
 \item My advisor Markus Borg~\orcid{0000-0001-7879-4371}.
 \item My brother Johan Bjäreholt and all the \href{https://activitywatch.net/contributors/}{ActivityWatch contributors}, for working with me all these years.
 \item The NeuroTechX crowd, specifically John Griffiths~\orcid{0000-0002-1764-2179} and Morgan Hough~\orcid{0000-0001-5256-413X}, for their support and time spent helping me.
 \item Pex Tufvesson and Carolina Bergeling at the Department for Automatic Control, for providing early guidance.
 \item Andrew Jay Keller at Neurosity, for gifting me a refurbished Notion DK1 to work with.
 \item Alex K. Chen, for referring me to all the right people.
 \item All the test subjects, for their time and interest.
 \item Everyone who has contributed to the open source tools I have used.
 \item Everyone who have supported me at LTH\@.
 \item Friends and family, for their neverending love and support.
\end{itemize}
}

    \keywords{electroencephalography, brain computer interfaces, time tracking, code comprehension, productivity, MSc}
\fi

\begin{document}

\makefrontmatter

% CS template inserts ToC as part of frontmatter
\ifx\UseCSLTH\undefined
    \tableofcontents
\fi

\listoftodos[Notes \& TODOs]
\vfill
\pagebreak

\begin{refsection}

\section*{Preface}

When I started university in 2013, I already knew that one day I wanted to work with brain-computer interfaces. During high school I had been soaked in transhumanist and hacker culture, spending more time online than in school.

I had come to the realization that the field was bound to fundamentally change how we interact both with computers and each other. However, I was aware that the field was in its earlier phases, and the work that needed to be done was outside of my expertise at the time.

So I asked how can I set myself up to be in a position to contribute to this impactful field of research. My answer at the time was to learn about data collection and analysis generally, and a couple years later applied my skills building ActivityWatch, as I figured the best approximation about what goes on in my head is what I'm looking at on my computer screen.

What I thought would be a small learning project soon took on a life of its own. People online started showing interest, and soon my brother joined in development. What started as a fun project to analyze my own device usage soon became a popular open source application with thousands of users. At some point it became apparent that once my Master's thesis was to be written, it would involve ActivityWatch one way or another.

As I was finishing up my final exams, I found an old paper in a moving box, it was a thesis proposal I had picked up at a career fair many years back. The proposal was from my advisor Markus Borg, and curious as I was, I contacted him and asked if it was still available. Markus enthusiastically replied, and helped me adapt his suggestion to combine it with my previous work on ActivityWatch.

As my thesis work progressed, I learned a lot about brain imaging and BCIs in general, and EEG in particular.

This thesis work has served as my first proper contribution to the field, all according to the plan I had made at the start of my university degree.

The Oxford English Dictionary defines `thesis' as ``a long essay or dissertation involving \emph{personal research}, written by a candidate for a university degree''. I can not think of more ``personal research'' than research in quantified self with personal data.

%\pagebreak

\section{Introduction}

\add[inline]{Write introduction}

%\section{Background}

People spend more time than ever using computing devices. Work, entertainment, and services, have been steadily moving online over the last few decades and this trend is expected to continue.
While several studies have been tracking how people spend time on their devices a wider study of how people's app usage is changing over time and how it varies with demographics, is not publicly available.

Furthermore, how different device activities affect the user behaviorally and neurologically is of interest for many areas of research, including:

\begin{itemize}
    \item psychological well-being, such as depression and social anxiety~\cite{selfhout_different_2009}\cite{shah_nonrecursive_2002}, stress~\cite{mark_stress_2014}, self-esteem, life satisfaction, loneliness, and depression~\cite{huang_time_2017}.
    \item the impact of screen time on children and adolescents~\cite{subrahmanyam_impact_2001}.
    \item attention span among media multitasking adults~\cite{mark_stress_2014}.
    \item enhancing personal productivity~\cite{kim_timeaware_2016}.
\end{itemize}

Understanding device use and the underlying cognitive processes are essential when designing for motivation, engagement and wellbeing in digital experiences~\cite{peters_designing_2018}.

This becomes especially relevant for knowledge workers, such as software developers, who spend the majority of their working time on computing devices.

\add[inline]{Mention of Quantified Self movement, and the applicability/usefulness of EEG data to the cause}

%\add[inline]{Add connection to software developers}

\subsection{Automated time trackers}

    Automated time-trackers have been developed for computing devices, with various applications such as tracking hours worked, personal productivity, managing excessive use of social networking sites (SNSs), and studying human behavior.

    \subsubsection{Commercial use}

        Companies like RescueTime~\cite{noauthor_rescuetime_nodate}, Hubstaff~\cite{noauthor_hubstaff_nodate}, and others offer automated time tracking as a service. These services let the user track their screen time by installing a program on their device which tracks the active application and sends the data to their servers for storage and analysis. The user can then view their data in a dashboard on the service's website. Some of these services, like RescueTime and Hubstaff, are marketed towards teams and professionals, who want to keep track of individual and team productivity.

        However, these services have some issues for use by researchers and individuals alike. Notably, their collection of detailed and non-anonymized behavioral data into a centralized system bring significant privacy concerns, especially in cases where the data is shared with a team or an employer.

        Other limitations of these services, such as low temporal resolution and limited event detail, cause additional issues for certain tasks that are timing-sensitive (such as ERPs), or preprocessing steps that can take advantage of high level of detail (like classifying activity).

    \subsubsection{Research use}

        Previous research has been published which used automated time trackers, such as TimeAware~\cite{kim_timeaware_2016} and ScreenLife~\cite{rooksby_personal_2016}. However, these previous contributions are --- like the commercial services --- not open source nor permissively licensed, and therefore not available for external research use nor further development.

    \subsubsection{ActivityWatch}

        The free and open source automated time tracker ActivityWatch~\cite{bjareholt_activitywatch_2020} addresses aforementioned issues with other software around source availability/licensing, privacy, temporal resolution, event detail, and cross-platform support.

        \todo[inline]{Update screenshot to v0.11}

        \begin{figure}[h]
        \centering
        \includegraphics[width=12cm]{img/screenshot-aw-activity.png}
        \caption{ActivityWatch activity dashboard. Showing top applications, window titles, browser domains, and categories.}\label{fig:aw}
        \end{figure}

        ActivityWatch as a project was started in 2016 by the author of this thesis, with his brother Johan joining development soon after. It has since become a popular open source alternative to other time tracking software. It has received numerous contributions from users\footnote{Contributor statistics available at \href{https://activitywatch.net/contributors/}{activitywatch.net/contributors/}} and is approaching 100,000 downloads\footnote{Download statistics available at \href{https://activitywatch.net/stats/}{activitywatch.net/stats/}}.

        \todo[inline]{What more could I add about ActivityWatch? There should be relevant stuff in the docs, and in the slides I made way back.}

\subsection{EEG and low-cost biosensors/functional brain imaging}

    Functional brain imaging methods such as fMRI, fNIRS, and EEG, have been used to study the relationship between cognitive or physical activity, and brain activity~\cite{floyd_decoding_2017}\cite{hong_classification_2015}\cite{fucci_replication_2019}. The more accurate methods such as fMRI are costly and inflexible/impractical for many uses.

    However, the recent availability of low-cost biosensors such as EEG, HEG, and fNIRS, enables studying brain activity during real-life tasks.

    But EEG is not without its limitations --- among them a notably low signal-to-noise ratio~\cite{mcfarland_eeg-based_2017}, as well as being difficult to use for discerning activity deeper in the brain~\cite{fahimi_hnazaee_localization_2020} --- yet these limitations have been overcome for many applications, like ERP experiments, and has even turned out prove sufficient for high-speed BCI applications through detecting visual evoked potentials (VEPs)~\cite{spuler_high-speed_2017}.

    To combat the low signal-to-noise ratio, machine learning methods have been employed with varying degrees of success. Examples from previous research include Convolutional Neural Networks (CNNs), which have been successful in classifying time series in general~\cite{zhao_convolutional_2017}, and EEG data in particular~\cite{schirrmeister_deep_2017}. As well as Hierarchical Convolutional Neural Networks (HCNNs), which have been used for EEG-based emotion recognition~\cite{li_hierarchical_2018}.

    \add[inline]{Separate sections for applications in BCIs vs neuroscience}

    \subsubsection{Applications in neurolinguistics}

        EEG has found many applications in neurolinguistics, both to understand how the brain processes natural languages as well as programming languages~\cite{prat_relating_2020}.

         As an example it has been shown that it is possible to classify if a participant is reading code or prose using fMRI~\cite{floyd_decoding_2017}, which has been replicated using EEG and low-cost biosensors~\cite{fucci_replication_2019}. The ability to distinguish between these two tasks have been has been explained neurologically by the recruitment of different neural networks in the brain~\cite{ivanova_comprehension_2020}. 

        The code vs prose comprehension task has also been modified into a writing task studied under fMRI, to further shed light to the underlying brain activity. In one study it was found that code and prose writing are significantly dissimilar tasks for the brain, where prose writing engages brain regions associated with language in the left hemisphere while while code writing engages brain regions associated with attention, working memory, planning, and spatial cognition in the right hemisphere\cite{noauthor_neurological_nodate}.

    \subsubsection{Applications in software engineering}

        \add[inline]{Applications to software engineers}

    % https://docs.openbci.com/citations

    % List of functional brain imaging techniques:
    %  - fMRI
    %  - fNIRS
    %  - EEG
    %  - HEG

\subsection{Aim of the thesis}

    The primary aim of the thesis is to improve upon previous attempts~\cite{fucci_replication_2019} to classify whether the user is reading code or prose using EEG data. This is to be achieved by using better EEG equipment and state of the art analysis methods such as Riemannian geometry. A secondary aim of the thesis is to investigate whether the ability of EEG analysis to classify code vs prose comprehension generalizes across more activities, such as the wide variety of tasks engaged in during organic device use.

    Secondary aims of the thesis include:

    \begin{enumerate}
        \item Implementing a classifier for device activities from EEG data, during organic device use
        \item Improving open-source tools for EEG analysis
    \end{enumerate}

    \add[inline]{Insert stuff from goal document}

\subsection{Related work}

    A systematic mapping study has been published which seeks to review the usage of psychophysiological data in software engineering~\cite{vieira_usage_2021}.

    It has previously been shown that fMRI~\cite{floyd_decoding_2017} and EEG\cite{fucci_replication_2019} provides enough information to classify whether a subject is reading prose or code. However, accuracy with single-channel EEG has been found to be poor, and notably outperformed by a heart rate variability (HRV) monitor.

    % Here, we used functional magnetic resonance imaging to investigate two candidate brain systems: the multiple demand (MD) system, typically recruited during math, logic, problem solving, and executive tasks, and the language system, typically recruited during linguistic processing.
    Recently, it has been shown that the multiple demand (MD) system is typically recruited for code comprehension tasks, as opposed to the language system that is typically recruited during prose comprehension~\cite{ivanova_comprehension_2020}. This sheds light on the significant differences in how the brain processes code vs prose.

    In addition to purely studying comprehension (reading) code and prose, an 2020 fMRI study showed that there are indeed significant neurological differences between \emph{writing} code and prose as well~\cite{noauthor_neurological_nodate}.

    In software engineerig research, low-cost biosensors such as wristbands that measure electrodermal activity and heart-related metrics have been used for emotion detection, as a tool in studying developer productivity.

    \add[inline]{Insert mention of preprint that Fucci mentioned?}

\chapter{Theory}

This chapter serves to introduce relevant theory on which the thesis rests. We will briefly review EEG, its use in research and practice. We also briefly review machine learning techniques used on EEG data, in particular Riemannian geometry.

\section{Electroencephalography (EEG)}\label{eeg-theory}

    % Intro
    Electroencephalography is a method used to measure the activity of neurons in the brain by recording the electrical potential of the scalp. These measurements are performed (sampled) hundreds of times per second, and are then put in sequence to form a signal of what can informally be referred to as ``brain waves''. 

    % Use in medicine & research
    As a non-invasive method it is widely used in medicine to diagnose and study a wide range of conditions, including epilepsy and sleep disorders. 
    It has also found use in research, in particular for studying \emph{Event Related Potentials}, or ERPs, which are stereotyped responses to a stimulus, an example of which can be seen in Figure~\ref{figure:n170}. Examples of common ERPs can be seen in Table~\ref{table:erps}.

    \begin{table}
        \centering
        \begin{table}
    \centering
    \begin{tabular}{ll}
        \toprule
        ERP & Elicited by
        \\
        \midrule
        N170 & Processing of faces, familiar objects or words.
        \\
        N400 & Words and other meaningful stimuli.
        \\
        P300 & Decision making, oddball paradigm.
        \\
        P600 & Hearing or reading grammatical errors and other syntactic anomalies.
        \\
        \bottomrule
    \end{tabular}
    \caption{Common \emph{Event-Related Potentials} (ERPs)}\label{table:erps}
\end{table}

        \caption{Common \emph{Event-Related Potentials} (ERPs)}\label{table:erps}
    \end{table}

    \begin{figure}
    \centering
    \includegraphics[trim=0 0 80 0, clip, width=14cm]{img/n170_viz.png}
    \caption{Example of aggregated epoch-averaged ERP trials from an N170 experiment, where the subject was shown either a face (in \textcolor{green}{\textbf{green}}) or a house (in \textcolor{red}{\textbf{red}}). The difference between the two stereotyped responses can be seen in black. The vertical line marks the beginning of the epoch.}\label{figure:n170}
    \source{eeg-notebooks \href{https://neurotechx.github.io/eeg-notebooks/auto_examples/visual_n170/01r__n170_viz.html}{experiment gallery} (BSD 3-clause)}
\end{figure}


    One difficulty of working with EEG lies in the low signal-to-noise ratio: measurements are on the order of microvolts, and what is being measured is the electrical activity of some subset of the approximately 86 billion neurons in the human brain. This often leads to signal artifacts from neural activity which is not under study, such as eye blinks and other muscle activity.

    It can be contrasted with \emph{electrocorticography} (ECoG), also known as \emph{intracranial electroencephalography} (iEEG), which is an invasive method where a craniotomy (a surgical incision on the skull) is performed to implant an electrode grid directly on the cerebral cortex. The result is higher spatial resolution, and improved signal-to-noise ratio due to closer proximity to neural activity. It is almost exclusively used in patients with intractable epilepsy (not responding to anticonvulsants).

    Electrodes can be placed at different locations on the scalp, targeting different regions of the brain. The system used to position the electrodes is called the 10–20 system (seen in Figure~\ref{fig:1020}), and is the standard way to label electrode placements. The \emph{10} and \emph{20} come from the fact that distances between adjacent electrodes are either 10\% or 20\% of the total front–back or right–left distance of the scalp.

    \begin{figure}[h]
        \begin{center}
            \includegraphics[width=8cm]{img/1020system.png}
        \end{center}
        \caption{The 10–20 system.}\label{fig:1020}
        \source{\href{https://commons.wikimedia.org/wiki/File:21_electrodes_of_International_10-20_system_for_EEG.svg}{Wikimedia Commons} (public domain)}
    \end{figure}

    For configurations with high-density or more specific electrode placement there is also the extended 10--10 system, using \emph{Modified Combinatorial Nomenclature} (MCN). In this system, the electrode placements are divided by 10\% increments instead of the usual 20\% (seen in Figure~\ref{fig:1010}).

    \begin{landscape}
        \begin{figure}
            \begin{center}
                \includegraphics[width=20cm]{img/1020system-extended-with-extra-info.png}
            \end{center}
            \caption{The extended 10–10 system, following the Modified Combinatorial Nomenclature (MCN).}\label{fig:1010}
            \source{\href{https://commons.wikimedia.org/wiki/File:EEG_10-10_system_with_additional_information.svg}{Wikimedia Commons} (CC-0)}
        \end{figure}
    \end{landscape}

    As measurements are taken on the scalp, neural activity of surface-level neurons (such as the cerebral cortex) are expected to dominate the signal, which makes it difficult to use when studying systems deeper in the brain (such as the hippocampus). As an example of this, eye blinks are easily identifiable in the signal when electrodes are placed on the frontal cortex (seen in Figure~\ref{fig:muselsl-signal}).

    %\add[inline]{Plots of PSD, PCA, etc}
    %\add[inline]{Discuss some of the review articles on EEG/BCI/ML}

    Raw EEG data usually contain artifacts from powerline noise, which needs to be filtered out. This is usually done with a simple bandpass filter, an example of which can be seen in Figure~\ref{fig:signal-unfiltered} and~\ref{fig:signal-filtered}.
    
    \begin{figure}[H]
        \includegraphics[width=14cm]{img/raw-signal-prefilter.png}
        \caption{Raw EEG signal with no filtering. Powerline noise clearly visible.}\label{fig:signal-unfiltered}
    \end{figure}

    \begin{figure}[H]
        \includegraphics[width=14cm]{img/raw-signal-postfilter.png}
        \caption{Same EEG signal as in Figure~\ref{fig:signal-unfiltered}, but with bandpass filter applied. We can see that powerline noise has been eliminated.}\label{fig:signal-filtered}
    \end{figure}

    After the signal has been filtered, one can perform spectral density approximation to estimate the relative power of different frequency bands in the signal. An example of frequency bands commonly used can be seen in Table~\ref{table:freq-bands}. We will come back to this later in the method, where we use this information as features for one of our classifiers.

    \begin{table}
        \centering
        \begin{tabular}{lll}
    \toprule
    Frequency band & Frequencies & Brain states \\
    \midrule
    Gamma ($\gamma$) & >35 Hz & Concentration \\
    Beta ($\beta$) & 16--35 Hz & Active, external attention, relaxed \\
    Sigma ($\sigma$) & 12--16 Hz & Sleep spindles \\
    Alpha ($\alpha$) & 8--12 Hz & Very relaxed, passive attention \\
    Theta ($\theta$) & 4--8 Hz & Deeply relaxed \\
    Delta ($\delta$) & 0.5--4 Hz & Sleep \\
    \bottomrule
\end{tabular}

        \caption{Characteristics of common frequency bands used in EEG research. Note the \emph{Sigma} band, which is usually grouped with the \emph{Beta} band, but is often used in sleep research where it targets sleep spindles during stage 2 NREM sleep. In some research, these bands are split further into slow and fast counterparts.}\label{table:freq-bands}
    \end{table}

    \begin{landscape}
        \begin{figure}
            \begin{center}
                \includegraphics[trim=60 50 50 60,clip,width=22cm]{img/muselsl-signal.png}
            \end{center}
            \caption{EEG signal viewed with muse-lsl. There are 3 eye blinks identifiable around $\SI{-4}{\second}$.\\ The signal has been bandpass filtered to eliminate noise.}\label{fig:muselsl-signal}
        \end{figure}
    \end{landscape}

\section{Machine Learning}

    Machine learning on EEG data utilizes several domain-specific methods, often similar to other methods for time-series data in general. However, some methods differ by taking advantage of the spatial information that is available through the analysis of interdependence between multiple EEG channels.

    Methods used in analysis and classification of EEG data include Linear Discriminant Analysis (LDA), Common Spatial Pattern (CSP) filters~\cite{barachant_common_2010}, spectral density estimation (bandpower features), and the computation of covariance matrices to estimate interdependencies between channels. 

    Furthermore, preprocessing methods include bandpass filtering, windowing, and various tricks for channel selection (in high-density setups). 

    With the use of these methods, we can preprocess the data, compute features, and train our classifiers.

    The underlying ML algorithms being trained on the features are often off-the-shelf logistic regression, support vector machines, random forests, etc. Some domain adaptations are found in more complex models like neural networks.

    %For logistic regression, the \emph{log loss} function $L$ is defined as:

    %\[ L(x, y') = \sum_{(x,y) \in D} -y \log y' - (1 - y) \log(1 - y')\]

    \subsection{Riemannian geometry}\label{section:riemannian-theory}

        %\add[inline]{Explanation of Riemannian geometry, from \href{https://colab.research.google.com/drive/1y9tq7-lJwusxtVgpB38y-p1pYw7hg0iu}{this tutorial we're working on}}

        EEG decoding approaches based on Riemannian geometry have led to state-of-the-art classification performance on many tasks. The idea behind it is to take advantage of the spatial structure of EEG, i.e.\ the fact that there are multiple channels which co-vary in specific ways.

        The Riemannian distance metric $\delta_G$ for two symmetric positive definite matrices $\mathrm{A}$ and $\mathrm{B}$ (such as covariance matrices) is~\cite{grafarend_metric_2003}:

        \[ \delta_G(\mathrm{A}, \mathrm{B}) = \sqrt{\sum_{i=1}^N \ln^2 \lambda_i (\mathrm{A}, \mathrm{B}) } \]

        Where $ \mathrm{\lambda}_i(\mathrm{A}, \mathrm{B}) $ are the eigenvalues from $|\mathrm{\lambda}\mathrm{A} - \mathrm{B}| = 0$.

        The naive Riemannian approach using \emph{Minimum Distance to Mean} (MDM) starts with computing the covariance matrix for each trial, and then estimating a mean covariance matrix for each class (the class centroid) using the Riemannian distance metric $\delta_G$. When classification is performed, the distance between the new covariance matrix and the class centroids is estimated using the Riemannian distance metric, and the new covariance matrix is classified according to which class centroid is closest.

        To allow for use of other classification approaches in the final step, such as logistic regression and SVMs, we can project our covariance matrices onto a tangent space (a schematic representation of this can be seen in Figure~\ref{figure:tangent-space}). Once projected onto this tangent space, our covariance matrices can be compared fairly well using the standard Euclidean distance metric, letting us use common methods which do not employ the Riemannian distance metric~\cite{congedo_riemannian_2017}.

        \begin{figure}[h]
    \centering
    \includegraphics[width=0.6\textwidth]{img/riemannian-tangent-space.png}
    \caption{Schematic representation of the symmetric positive definite matrix manifold, the geometric mean $G$ of two points and the tangent space at $G$. The geometric mean of these points is the midpoint on the geodesic connecting $C_1$ and $C_2$, i.e.\ it minimizes the sum of the two squared distances. The map from the tangent space to the manifold is an exponential map. The inverse map is a logarithmic map.}\label{figure:tangent-space}
    \source{Congedo et al.~\cite{congedo_riemannian_2017}}
\end{figure}


    \begin{comment}
    \subsection{Neural networks}
        Many state-of-the-art models in the field of EEG are deep-learning based models, usually employing convolutional neural networks (CNNs).
    \end{comment}

\chapter{Method}

Our method starts with gathering the data, and to get started with data collection we configure the equipment and tools needed to perform the experiment.

Once our data collection is done, we continue with our analysis by training a classifier using various machine learning methods.

\includegraphics[width=10cm]{img/method.png}

\section{Collection}

In order to perform our experiment and analysis, we need two types of data: EEG data, and device activity data to label the EEG data.

    \subsection{Collection of EEG data}

        EEG data was collected during organic device use and under controlled conditions.

        For both conditions, code from the open source eeg-notebooks~\cite{barachant_eeg-notebooks_2020} was adapted to record the raw EEG stream into a CSV file.

        Depending on the device used we require certain software to connect to the devices. We used muse-lsl for the Muse S~\cite{muse-lsl} which in turn uses Lab Streaming Layer. To support OpenBCI and Neurosity devices we used brainflow~\cite{noauthor_brainflow_2020}.

        \subsubsection{During organic device use}

            For the organic device use conditions, we primarily used the Muse S due the superior comfort and ease of use compared with the alternatives, making it especially suitable for long recordings.\footnote{A wet electrode cap system was also considered, but ultimately not investigated due to being inconvenient to setup.}

            The subject was then simply asked to go about their usual device activities, often consisting of a mix of work (email, writing prose, writing code) and leisure (watching YouTube, reading Twitter).

        \subsubsection{During code vs prose comprehension task}

            For the controlled condition, we ended up using the Muse S as well due to the comfort and ease of setup.

            \todo[inline]{Update with latest}
            We had 9 subjects, sampled by convenience. They were mostly male in their late 20s.

            We implemented the task in eeg-notebooks~\cite{barachant_eeg-notebooks_2020}, which uses previously mentioned libraries for data collection as well as PsychoPy~\cite{peirce_psychopy2_2019} to provide the stimuli.

            The experiment consists of presenting images with code or prose comprehension tasks, as seen in Figure~\ref{fig:codetask} and~\ref{fig:prosetask}.

            \begin{figure}[h]
                \begin{center}
                    \includegraphics[trim=0 120 0 0,clip,width=100mm]{img/final-1-1.png}
                \end{center}
                \caption{Sample of the code comprehension task}\label{fig:codetask}
            \end{figure}

            \begin{figure}[h]
                \begin{center}
                    \includegraphics[width=100mm]{img/bugs_1.PNG}
                \end{center}
                \caption{Sample of the prose review task}\label{fig:prosetask}
            \end{figure}

            Before each run, the subject was asked about their gender, age, and software development experience (specifically experience with C/C++). For good measure, we also asked if the subject had consumed caffeine the hours prior to the experiment.

            After these questions we put the devices on and ensure we get good signal by inspecting it in real time with the viewer provided by muse-lsl. The viewer itself does simple bandpass filtering between 3--40Hz, and the signal quality is indicated by the standard deviation of the filtered signal.

        \subsubsection{Devices}

            We experimented with several devices but eventually settled on the Muse S. The motivation for choosing the Muse S was mainly due to comfort and ease of use. The other devices considered included the OpenBCI Cyton (with Ultracortex headset), and the Neurosity Crown.\footnote{Earlier in the work, before we received the Crown, we were also generously gifted a Neurosity Notion DK1 to get a head start.}

            The Muse S is a 4-channel EEG headband with electrodes at TP9, AF7, AF8, and TP10, with the reference electrode at Fpz~\cite{krigolson_choosing_2017}.\footnote{According to the 10--20 system.}

            %\begin{minipage}

            %\begin{widepage}
                \begin{table}[h]
                \centering
                \begin{tabular}{llccc}
                    \toprule
                    Manufacturer
                    & Device
                    & Channels
                    & Sampling rate
                    & Comfort
                    \\
                    \midrule
                    InteraXon
                    & Muse S
                    & 4
                    & 250Hz
                    & High \\
                 OpenBCI
                    & Cyton (with Ultracortex)
                    & 8
                    & 125--250Hz
                    & Low \\
                 % generously gifted by Neurosity
                 %Neurosity
                 %   & Notion DK1
                 %   & 8
                 %   & 250Hz
                 %   & Medium \\
                  % preordered, arrives in late spring
                  Neurosity
                    & Crown
                    & 8
                    & 250Hz
                    & Medium \\
                    \bottomrule
                \end{tabular}
                \caption{Devices used}\label{table:devices}
            \end{table}
            %\end{widepage}

            %\begin{center}
                \begin{figure}[H]
                \centering
                %\begin{widepage}
                \begin{tabular}{c}
                    \includegraphics[width=80mm]{img/Muse-S.jpg}
                    \\
                    (a) Muse S
                    \\[6pt]
                    \includegraphics[width=80mm]{img/openbci-cyton.jpg}
                    \\
                    (b) OpenBCI Cyton with Ultracortex
                    \\[6pt]
                    \includegraphics[trim=0 100 0 0,clip,width=100mm]{img/crown-1.png}
                    \\
                    (c) Neurosity Crown
                    \\[6pt]
                \end{tabular}
                \caption{Photos of devices used}
                %\end{widepage}
            \end{figure}
            %\end{center}

            %\end{minipage}

    \subsection{Collection of device activity data}

        All device activity is collected using the automated time tracker ActivityWatch~\cite{bjareholt_activitywatch_2020-1}.

        ActivityWatch collects data through modules called watchers which report to the ActivityWatch server. It comes with two watchers by default:

        \begin{itemize}
            \item aw-watcher-window, tracks the active window and its title
            \item aw-watcher-afk, tracks if the user is active or not by observing input device activity
        \end{itemize}

        We've also built a custom watcher, aw-watcher-input, to track metrics of mouse and keyboard activity. It tracks by listening to mouse and keyboard events and records the distance\footnote{in pixels} the mouse moves and number of clicks (but not which key was clicked). Every second this is bundled into an event, the values are reset, and then it continues with the next event. It was inspired by similar functionality in Andrej Karpathy's ulogme~\cite{karpathy_ulogme_2016}.

        A limitation that we have to consider is that the window watcher uses a polling method to track the active window, with a default poll time of 1 second. This means that we can't rely on the timestamps to mark the exact time the window became active/inactive.

        The data from ActivityWatch is processed and categorized such that the resulting data has the 3 columns \mintinline{python}{start, end, category}. The category is determined by a regular expression that matches on window titles and URLs, such as \mintinline{python}{github.com}.

\section{Analysis}

    For classification and analysis, we used common open source Python libraries for data analysis, like numpy~\cite{harris2020array}, pandas~\cite{reback2020pandas}, and scikit-learn~\cite{scikit-learn}. In addition, we used less common libraries tailored specifically for working with EEG data, such as MNE~\cite{noauthor_mne-python_2020}, pyriemann~\cite{alexandre_barachant_2020_3715511}, and YASA~\cite{vallat_yasa_2020}.

    \subsection{Labelling}
        For the uncontrolled condition, we split the EEG data into epochs using the categories assigned by our ActivityWatch script.

        For the controlled experiment, we split the EEG data into epochs using the trial markers, resulting in one epoch per stimuli.

    \subsection{Data transformation}

        In order to train on the variable-length epochs, we need to split each epoch into a fixed-duration window, which can then we use to train and classify our model.

        Dimensions of each epoch matrix: \[ (n_{samples}, n_{channels}) \]

        Where $n_{samples}$ is the total number of samples for the epoch (variable-length), and $n_{channels}$ is 4 for the Muse S.

        Since the matrix has variable dimensions for each epoch, we split it into $\sim$5s windows, which at the 256Hz sampling frequency of the Muse gives us 1280 samples per window.

        Dimensions of the window matrix: \[ (n_{windows}, n_{channels}, 1280) \]

        We experimented with different windowing methods to potentially augment the data...  \todo[inline]{...and? Should a sliding window approach be used to improve data augmentation?}

    \subsection{Data cleaning}

        % TODO: Should this say 'reject windows' instead?
        We reject samples that either:

        \begin{enumerate}
            \item Don't have an assigned class
            \item Have a bad signal quality (as indicated by a high signal variance)
            \item Are too short (due to missing samples)
        \end{enumerate}

        \todo[inline]{List how many epochs are rejected by each cleaning step}

    \subsection{Feature engineering}

        One approach to classifying EEG data is to perform feature extraction/engineering. Common features used for EEG data include bandpower ratios, as well as covariance matrixes using the Riemannian metric.

        To evaluate most common machine learning classifiers, we need to

        \subsubsection{Bandpower}

            \todo[inline]{Refer/move to theory section instead?}

            Bandpower features are simple and commonly used in EEG research for many tasks, including the paper by Fucci et al we seek to improve upon~\cite{fucci_replication_2019}. As a reference, we implemented classifiers which solely used bandpower features as input, to gain information of how much any improvement from classifier performance is likely due to better EEG equipment versus how much is due to from improved analysis methods.

            To compute this feature, we utilized the bandpower function provided by YASA~\cite{vallat_yasa_2020}. The implementation estimates the power spectral density using Welch's method for each channel, and bins them by their associated frequency band.

            To further enrich our feature vector, we can use ratios between two frequency bands.

        \subsubsection{Riemannian geometry}

            \todo[inline]{Refer/move to theory section instead?}

            The \improvement{according to whom?}{state of the art in many EEG classification tasks} involves the use of Riemannian geometry. For this, we used the open source pyriemann library by Alexandre Barachant\footnote{First author of the original paper to apply Riemannian geometry to EEG~\cite{barachant_classification_2013}}.

    \subsection{Neural Networks}

        One of the classifiers we want to train is a neural network. We use braindecode~\cite{schirrmeister_deep_2017}\cite{noauthor_braindecode_2021}, a neural network toolbox for EEG data that uses PyTorch and integrates it with scikit-learn through skorch.

        The networks provided by braindecode are convolutional\ldots

    \subsection{Cross Validation}

        We use LORO (``Leave-One-Run-Out'') cross-validation, a variation of LOGO (``Leave-One-Group-Out''), in order to ensure the samples used in validation are using subjects or tasks that are unseen in training.

        We attempt both out-of-subject validation and out-of-task validation in order to estimate the ability of the classifiers to generalize across subjects and tasks.

    \subsection{Single subject}

        % TODO: what experiments?
        % TODO: what devices?
        % FIXME: is this out of scope?
        We experimented with single-subject analysis to validate different devices and tasks.

\chapter{Results}\label{section:results}

    We present the results from our two different experiments, and compare the results of our code vs prose experiment to the original study~\cite{floyd_decoding_2017} and the replication study~\cite{fucci_replication_2019}. 

    In addition, we investigate the results from our naturalistic device use experiment to see if results from the code vs prose experiment generalizes to other types of device activity.

    \vfill
    \pagebreak
    \section{Code vs prose task}
        Table~\ref{table:bac-all} and~\ref{table:bac-selective} shows the performance we achieved for the code vs prose task, for two different subject selections. Figure~\ref{fig:timebars} shows a detailed overview of the data, and classification results for one example subject-fold. Finally, we compare our results to previous studies in Table~\ref{table:compare-results}.

        

        Our top-performing classifier, using Riemannian methods and cross-validated using LORO, yields a median balanced accuracy of $0.749$  for window-level classification and $0.9$ for epoch-level classification (seen in Table~\ref{table:bac-selective}).

        The \textbf{Bandpower} columns show the results for each subject-fold using the bandpower benchmark for window-level data and epoch-level data, respectively. Correspondingly, the \textbf{Riemannian} columns show the results using Riemannian geometry. The classifier using Riemann geometry tends to outperform the baseline in each fold.

        We do window-level classification by training on \SI{5}{\second} windows (as described in Section~\ref{section:transform}).

        We achieve epoch-level classification by training a window-level classifier just as for the \SI{5}{\second} windows, we then make a classification for the entire epoch by taking the mean of the prediction probabilities from the windows in that epoch.

        \begin{table}[h]
            \centering
            \begin{tabular}{lcccc}
                \toprule
                & \multicolumn{2}{c}{\textbf{Riemannian}} & \multicolumn{2}{c}{\textbf{Bandpower}} \\
                \cmidrule(lr){2-3}
                \cmidrule(lr){4-5}
                \textbf{Subject} & Window-level & Epoch-level & Window-level & Epoch-level \\
                \midrule
                \#0  & 0.608 & 0.603 & 0.502 & 0.551 \\
                \#1  & 0.802 & 0.864 & 0.666 & 0.677 \\
                \#5  & 0.589 & 0.534 & 0.602 & 0.667 \\
                \#6  & 0.701 & 0.767 & 0.702 & 0.675 \\
                \#7  & 0.694 & 0.800 & 0.725 & 0.733 \\
                \#8  & 0.547 & 0.542 & 0.546 & 0.625 \\
                \#9  & 0.484 & 0.500 & 0.418 & 0.500 \\
                \#10 & 0.474 & 0.500 & 0.452 & 0.500 \\
                \midrule
                Median & 0.5985 & 0.573 & 0.574 & 0.646 \\
                \bottomrule
            \end{tabular}
            \caption{The balanced accuracy for each LORO subject-fold. Excluding subjects \#3 and \#4. Subject \#2 does not exist due to an off-by-one mistake during experiment setup (what should have been subject \#2 became \#3).}\label{table:bac-all}
        \end{table}

        In Table~\ref{table:bac-all} we see that performance is bad (no better than chance, i.e. $BAC \approx 0.5$) for several subjects. We investigate these and find issues with the quality and amount of data (having no predicted samples for one of the classes). Due to this we remove them from analysis, and get the improved results seen in Table~\ref{table:bac-selective}. We discuss our subject selection further in Section~\ref{section:discussion}.

        Compared to previous studies, we achieve a moderate improvement over the EEG-only classifier trained in Fucci et al., and achieve a similar performance to the fMRI study by Floyd et al. (seen in Table~\ref{table:compare-results}).

        \begin{table}[h]
            \centering
            \begin{tabular}{lcccc}
                \toprule
                & \multicolumn{2}{c}{\textbf{Riemannian}} & \multicolumn{2}{c}{\textbf{Bandpower}} \\
                \cmidrule(lr){2-3}
                \cmidrule(lr){4-5}
                \textbf{Subject} & Window-level & Epoch-level & Window-level & Epoch-level \\
                \midrule
                \#0 & 0.673 & 0.727 & 0.511 & 0.541 \\
                \#1 & 0.895 & 0.955 & 0.689 & 0.809 \\
                \#5 & 0.616 & 0.542 & 0.628 & 0.750 \\
                \#6 & 0.864 & 0.908 & 0.739 & 0.737 \\
                \#7 & 0.749 & 0.900 & 0.733 & 0.733 \\
                \midrule
                Median & 0.749 & 0.900 & 0.689 & 0.737 \\
                \bottomrule
            \end{tabular}
            \caption{The balanced accuracy for each LORO fold/subject. Excluding subjects 3, 4, 8, 9, and 10.}\label{table:bac-selective}
        \end{table}

        It should be noted that the epoch-level numbers are highly variable due to the small number of subjects (for the median) and total trials (for each subject). Therefore, we only present our best window-level results in Table~\ref{table:compare-results}.

        \begin{table}[h]
            \begin{center}
                \begin{tabular}{lcccc}
                    \toprule
                    & \multicolumn{2}{c}{\textbf{This study}} & \multirow{2}{*}{\textbf{Fucci et al.}} & \multirow{2}{*}{\textbf{Floyd et al.}} \\
                    \cmidrule(lr){2-3}
                    & Riemannian & Bandpower & & \\
                    \midrule
                    Overall & 0.75 & 0.69 & 0.66 & 0.79 \\
                    \bottomrule
                \end{tabular}
                \caption{Result comparison between the previous studies and this study. Best balanced accuracy scores are reported. For this study, we used the best window-level score. For Fucci et al.\ we chose the best EEG-only score.}\label{table:compare-results}
            \end{center}
        \end{table}

        \begin{comment}
            \begin{table}
                \begin{center}
                    \begin{tabular}{lcc}
                        \toprule
                                & Window-level & Epoch-level \\
                        \midrule
                        Precision & 74.7\% & 85.4\%  \\
                        BAC       & 69.6\% & 76.7\%  \\
                        \bottomrule
                    \end{tabular}
                    \caption{Performance statistics of our models trained on all subjects with good signal quality except number \#6, which is used for testing.}\label{fig:stats}
                \end{center}
            \end{table}
        \end{comment}

        % Keep this?
        \begin{comment}
            The rows in the figure can be interpreted as follows:

            \begin{itemize}
                \item Image: the stimuli image shown.
                \item Label: the class of the stimuli.
                \item Predicted: the predicted class.
                \item Correct: whether the prediction matches the label.
                \item Subject: the subject performing the task.
                \item Split: the train/test split.
                \item Quality: whether the signal meets our quality standard.
            \end{itemize}
        \end{comment}

        \begin{landscape}
            \begin{figure}
                \centering
                \includegraphics[width=24cm]{img/timebars.png}
                \caption{Visualization of the labeled data with classifications from one example subject-fold. Shows the \emph{Image} (stimuli), the class \emph{Label} for that stimuli (\textcolor{NavyBlue}{\textbf{blue}} is code, \textcolor{BurntOrange}{\textbf{orange}} is prose), the \emph{Predicted} class, whether the prediction is \emph{Correct}, the \emph{Subject}, the \emph{Split}/Fold (\textcolor{NavyBlue}{\textbf{blue}} shows the training set, \textcolor{Goldenrod}{\textbf{yellow}} the test set), and our threshold measure for signal \emph{Quality} (\textcolor{Green}{\textbf{green}} indicates acceptable quality). The x-axis is the window index, sorted by acquisition time.
                \\
                \\
                It can be seen that (1) subjects \#3 and \#4 have bad signal quality, and have therefore been excluded from the training set. (2) The subjects \#9 and \#10 have also been excluded from training due to issues during data collection. (3) For subject \#1 the stimuli images were not shuffled. (4) Subject \#0 appears twice, as they did two sessions (using unseen stimuli).}\label{fig:timebars}
            \end{figure}
        \end{landscape}

        \begin{comment}
            \begin{figure}[h]
            \centering
            \includegraphics[width=12cm]{img/roccurve.png}
            \caption{Receiver operating characteristic (ROC) curve for subject \#6.}\label{fig:roc}
            \end{figure}
            \change[inline]{Update with higher-res image}
        \end{comment}

    \section{Naturalistic device activity}

        As described in Section~\ref{section:collect-eeg-naturalistic}, we collected approximately \SI{5}{hours} of labeled EEG data using our labels described in Section~\ref{section:collect-usage}. 

        The data was collected on several different days, with a breakdown of the date and class distribution shown in Figure~\ref{figure:dayclass-dist}. Using that data, we trained classifiers for each pair of label combinations. 

        Our results are seen in Table~\ref{table:scores-natural}.

        %\add[inline]{Förklara vad kolumnerna i löptexten så får du lite textvolym. Kan du säga något om när data samlades in? Var det vid ett tillfälle? Flera tillfällen? Tid på dygnet? Eventuellt kan någon sådan metadata även leda till en figur som illustrerar något... Lite deskriptiv statistik kanske finns också? Hur många tillfällen spelades in? Vad var medelvärdet på inspelningstid? Hur såg signalstyrkan ut? Vad triggade dig att sluta spela in? Slut på batteri? Annat? (obekväm sensor, störd av något, eller kände dig färdig)}

        \begin{table}[h]
            \centering
            \begin{tabular}{llrr}
                \toprule
                \textbf{Experiment} & \textbf{Score} & \textbf{Support} & \textbf{Hours} \\
                \midrule
                Programming vs Writing & 0.676 & (1386, 209) & 2.22h \\
                Programming vs Twitter & 0.695 &  (1386, 949) & 3.24h \\
                Programming vs YouTube & 0.672 &  (1386, 266) & 2.29h \\
                Twitter vs Writing & 0.833 &  (949, 209) & 1.61h \\
                Twitter vs YouTube & 0.604 & (949, 266) & 1.69h \\
                YouTube vs Writing & 0.889 &  (266, 209) & 0.66h \\
                \bottomrule
            \end{tabular}
            \caption{The scores for each label pairing. The \textit{Score} is the mean balanced accuracy of the StratifiedKFold splits. The \textit{Support} is the number of windows for each class. \textit{Hours} is the sum of both classes' duration.}\label{table:scores-natural}
        \end{table}

        \begin{figure}[h]
            \centering
            \includegraphics[width=12cm]{img/naturalistic-dayclass-dist.png}
            \caption{Class and date distribution of collected data.}\label{figure:dayclass-dist}
        \end{figure}

        \vfill

\chapter{Discussion}\label{section:discussion}

We now move on to discussing our results and answering our two research questions. We will also discuss some threats to the validity, potential applications, ethical considerations, and a brief mention of research initiatives we have contributed to along the way.

\section{RQ1 --- Improving upon previous results in classifying code vs prose}

We found that we do improve upon the best classifier performance by Fucci et al.\ in their EEG-only configuration~\cite{fucci_replication_2019}. However, we had to remove certain subjects from our sample, and our sample in general was smaller and more homogenous.\footnote{The reason for this cherry-picked subject selection is due to issues with signal quality and limited data (due to spurious device disconnects).}

Our findings also show that Riemannian methods, as used in this thesis, are better at distinguishing between the code and prose tasks than the bandpower-features approach, as used by Fucci et al. Our Riemannian score\footnote{Scores taken from Table~\ref{table:compare-results}} was $0.75$, which is better than both our bandpower-features score of $0.69$, as well as the Fucci et al score of $0.66$. This is in line with our expectations, as Riemannian methods are considered state-of-the-art for many EEG and BCI tasks, unlike bandpower features.

We were not able to outperform the results by Floyd et al.~\cite{floyd_decoding_2017}. This is not surprising given they were using fMRI, which has far superior spatial resolution and can therefore accurately detect elevated activity in specific brain regions over the duration of an epoch. However, our method using EEG has the benefit of high temporal resolution, allowing classification using small amounts of data, enabling near real-time applications.

\section{RQ2 --- Identify software developers' work tasks based on brain activity}

We train several classifiers and achieve promising results. We were especially successful in distinguishing work (writing code) from social media (Twitter), with a BAC of $0.69$.

Among the classifiers we train, we found that discerning leisure-activities from each other (such as YouTube vs Twitter, with a BAC of $0.60$) is harder than discerning work from leisure (such as writing code vs Twitter). We also find that EEG is sufficient to not only pick up differences in code vs prose \emph{comprehension} but also in \emph{writing}, as shown by our writing code vs writing prose categories (BAC of $0.68$).

We conclude that our preliminary results indicate that it is possible to discern several different device activities using EEG\@. However, more data from multiple subjects are needed in future work to validate the method and results. We also want to note that we only try to discern two types of activity at a time, pair by pair, and do not try to identify them among all possible work activities (a much more difficult task, left for future work).

\section{Threats to validity}\label{section:threats}

    During our work we have considered several potential threats to validity. Some of these arise from our limited and biased dataset, while others are about the task and methodology.

    Starting with our dataset, we collected on mostly right-handed males in their late 20s. This uniform/homogenous sample may lead to less data needed to train a classifier for that particular group, but does not necessarily generalize well to the population at large. Future studies should include a more diverse sample of participants.

    We also had issues during data collection with spurious disconnects from the device, leading to data loss and incomplete experiment runs. This is a threat to the validity of the study due to not all subjects having undergone as many (or the same) trials.

    Among our considerations, one threat to the validity is the stimuli images themselves (seen in Figure~\ref{fig:tasks}). In Floyd et al.~the images used for prose comprehension are in fact more like a code review task, while Fucci et al.~modified the images to test comprehension, instead of ability to judge correctness. We also discovered that subjects found the prose stimuli used by Floyd et al.~confusing, and it would have been preferable to use prose stimuli more like that used by Fucci et al.

    The stimuli images differ on more than just content. Examples of such differences are the background color, the difference in eye saccades while reading (eyes jump around more during the code tasks, where the user may have to jump between the code and the question about it). We have not been able to discard the possibility that the front-heavy electrode placements (with reference electrode at Fpz) lead to much of the signal being from eye saccades. 

    Future research could evaluate this further either by using an eye tracker or by using an EEG device with different placements of both the electrodes and reference electrode.

    %\add[inline]{Investigate threats to validity mentioned by previous authors}

\section{Applications to software engineering}

Gaining insight into the minds of developers at work can be used to aid and enhance the productivity of developers in several ways. As an example, it could be used to assist the developer in real time, like helping to identify developer confusion, but other applications could be imagined where a summary of the developers' brain activity during a particular commit (such as if the developer was confused, focused, tired) is included in the commit message, to help identify commits prone to introducing bugs.

%\add[inline]{Include mention and image of mood attached in commit message from Fucci talk?}

Applications of our results include:

\begin{itemize}
    \item Use the confidence in the task prediction as an alternative measurement of focus (not merely measuring that the subject was focused, but what they were focused on). 
    \item Detecting distraction.
    %\item The ability to collect all data needed to train an EEG classifier during normal device use.
\end{itemize}

We also consider our novel idea of `mental churn', a neurally integrated alternative to the measure of code churn, where not only code changes are measured, but also the \emph{attention} a file or piece of code receives as a whole. It can be viewed as quantifying the \emph{cognitive load} of the developer, and attributing it to the current context. This notion of mental churn could also include what response the code elicited in an engineer (confusion, focus, confidence).

\section{Ethical considerations}

    When studying EEG data a range of ethical considerations arise. 

    \begin{itemize}
        \item Could the data be considered personally identifiable information (PII) and thereby fall under GDPR's regulations concerning ``sensitive user data''? 
        \item How privacy sensitive are EEG recordings? Could they contain something the subject would rather keep private? (could have medical implications)
        \item How do we build large public datasets, while preserving participants privacy?
        %\item How can one use EEG in a work environment without unreasonable surveillance of the employee?
    \end{itemize}

    Companies such as Neurosity have taken an approach with their products where all the processing happens on-device, and only aggregates and classifier outputs are sent to the cloud for storage and presentation to the user. This is similar to the approach taken by ActivityWatch for device usage data, where all data is stored locally and never sent to a remote server for processing. 

    We believe this approach is the most privacy-preserving, but it comes at the cost of difficulty in building large datasets, as the data is no longer collected in a central repository under the control of companies or service providers. Therefore, there is a need to bridge the gap between privacy and sharing data.

    Simple solutions to this problem include opt-in to data collection, which is easy to do at scale for companies offering EEG devices. But other efforts include projects such as the NeuroTech Challenge (mentioned in Section~\ref{section:ntcs}).

    There are also more advanced solutions, such as privacy-preserving ML systems. One example of such solutions is \href{https://github.com/OpenMined/PySyft}{PySyft}, which enables private deep learning using federated learning, differential privacy, and encrypted computation (through Multi-Party Computation and Homomorphic Encryption) to work within common deep learning frameworks such as PyTorch and Tensorflow.

\section{Democratization of neuroscience}

    This thesis was made possible due to the efforts of individuals and communities such as NeuroTechX to democratize neuroscience. Indeed, it is the explicit goal of the NeuroTechX eeg-notebooks project to `democratize the neuroscience experiment'. Combined with the rapid cost reduction of research-grade EEG equipment over the last decade it has enabled hobbyists to design and perform high-quality neuroscience experiments. This enables citizen scientists to contribute to data collection, an example of such an effort is the NeuroTech Challenge Series (described in Section~\ref{section:ntcs}).

    As development of BCIs advance and the consumer market for EEG devices grow (as evidenced by new devices being released with a regular cadence by InteraXon and Neurosity) we expect to see more uses and applications of these devices.

    Much of this work was made possible due to the efforts of communities such as NeuroTechX to democratize neuroscience by publishing tools for running experiments. As part of the thesis, we have contributed changes back to some of the tools used (as mentioned in~\Vref{section:aim}).

\subsection{Crowdsourcing data}

    Collecting data is a significant time sink for researchers, and efforts to crowdsource data from the general public are difficult for EEG as it still requires access to the equipment, the knowledge to operate it, as well as considerations like signal quality, electrode placement, and other factors that might invalidate the data.

    Crowdsourcing data comes with new challenges. One of them is that data is now recorded by many different devices, with differences in channel count, sampling rate, electrode placement, and so on, that are difficult to combine in the same dataset. 

    A potential solution to the problem of learning new variations from a small additional sample is to attempt cross-task or cross-subject transfer learning. In such a setup, a model is already trained on a group of subjects, or a similar task, and through a small amount of new data for a particular subject or task, the model can adapt to learn those previously unseen subjects or tasks. This is discussed further in Section~\ref{section:transfer-learning}.

    %\paragraph*{NeuroTech Challenge Series}

    As part of the thesis work I have contributed to the \href{https://neurotech-challenge.com/}{NeuroTech Challenge Series} (NTCS), an effort in crowdsourcing EEG data using the experiments built in \texttt{eeg-notebooks}. The challenge can be performed by anyone with access mobile consumer EEG devices, like those from Muse, OpenBCI, and Neurosity\@. The project is a collaborative effort led by John Griffiths at the University of Toronto, with support from OpenBCI and NeuroTechX.\label{section:ntcs}


\section{Transfer learning}\label{section:transfer-learning}

An important aspect, as highlighted earlier in this thesis, is the ability of a classifier to be able to work on subjects unseen in training. Some BCI systems employ a calibration phase to achieve greater performance, but this can be time-consuming and straining for the user.

To minimize this need for calibration is a stated research goal in Khazem et al.~\cite{khazem_minimizing_2021}, which presents what is called Riemannian Transfer Learning. They build on Riemannian geometry used in state-of-the-art classifiers, and develop a variation of MDM (explained in Section~\ref{section:riemannian-theory}) called \emph{Minimum Distance to Weighted Mean} (MDWM). The method takes a parameter $\lambda$ (where $0 \leq \lambda \leq 1$) that controls how much the algorithm should rely on the class centroids learned from past subjects versus the calibration data from a new subject. This is useful in online learning contexts, where the parameter can initially start at $0$ and then be incrementally adjusted towards $1$. The researchers found $\lambda = 0.7$ to be a reasonable value for many tasks. 

The researchers also considered using weights for each source subject as done by Kalunga et al.~\cite{kalunga_transfer_2018}, which could be used to adjust for subject similarity.

We investigated this avenue of inquiry to potentially minimize the amount of data collection needed, but in the end did not have the time to implement it.

\chapter{Conclusions}

Our results successfully replicate previous research, showing that it is possible to distinguish between reading code and prose using EEG, and improves upon the state of the art in this regard by achieving a roughly $\sim$75\% balanced accuracy (using CV-LORO). However, the reader should note limitations of our study in Section~\ref{section:threats}.

Furthermore, our naturalistic experiments indicate it is possible to distinguish many other device activities from each other using consumer-grade EEG devices. Among these some data seems to suggest that EEG is sufficient to not only pick up differences in code vs prose \emph{comprehension} but also in \emph{writing}.

\section{Future work}

Future work could be to integrate the codeprose task into \texttt{moabb} to make it easier to replicate the results and evaluate new methods. As mentioned, the study would also benefit from more data collection, possibly through the NeuroTech Challenge Series (mentioned in Section~\cite{section:ntcs}).

As mentioned in the method, this study uses prose \emph{review} images from Floyd et al., as opposed to the prose \emph{comprehension} images (in Italian) used by Fucci et al. Ideally, one should create an english prose comprehension set of stimuli images, similar to the ones used by Fucci.



% References
%\bibbysegment{}
\printbibliography[category=cited]

% Further reading (uncited)
\nocite{*}
\defbibenvironment{bibnonum}
  {\list{}
     {\setlength{\leftmargin}{\bibhang}%
      \setlength{\itemindent}{-\leftmargin}%
      \setlength{\itemsep}{\bibitemsep}%
      \setlength{\parsep}{\bibparsep}}
  }
  {\endlist}
  {\item}
\printbibliography[notcategory=cited, env=bibnonum, heading=notcited]

\end{refsection}
\end{document}
