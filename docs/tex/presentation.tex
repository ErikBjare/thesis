\documentclass[xcolor={dvipsnames,table},12pt]{beamer}

\usepackage[smartEllipses]{markdown}  % For markdown
\def\markdownOptionOutputDir{build}  % Needed, see https://github.com/Witiko/markdown/issues/6#issuecomment-328699108

\usepackage{array}
\usepackage{hyperref}
\usepackage{makecell}
\usepackage{comment}
\usepackage{xargs}
\usepackage{verbatim}
\usepackage{subfig}
\usepackage{booktabs}
\usepackage{changepage}
\usepackage{rotating, graphicx}
\usepackage{multirow}
\usepackage{pdflscape}
\usepackage{afterpage}
\usepackage{varioref}
\usepackage[export]{adjustbox}
\usepackage[dvipsnames]{xcolor}
\usepackage[bottom]{footmisc}
\usepackage[colorinlistoftodos, prependcaption, textsize=tiny]{todonotes}

% References
\usepackage[backend=biber, style=numeric, sorting=none]{biblatex}
\usepackage{cleveref}

% Encoding and languages
\usepackage[utf8]{inputenc}
\usepackage[english]{babel}
\usepackage[T1]{fontenc}        % För svenska bokstäver
%\usepackage[swedish]{babel}    %Svenska skrivregler och rubriker

% Graphics
\usepackage{epsfig}
%\usepackage[dvips]{graphics}

% For adding source captions to figures.
% From: https://tex.stackexchange.com/a/246285/36302
\newcommand{\source}[1]{\vspace{-0.3cm} \caption*{\footnotesize{Source: \textit{{#1}}}}}

% Code snippets
\usepackage[outputdir=build]{minted}
\usemintedstyle{vs}

% TODO: Move to commands file
\newcommandx{\orcid}[1]{\href{https://orcid.org/#1}{\includegraphics[width=0.7em]{img/orcid-icon.png}}}

\def\mytitleen{Classifying brain activity using electroencephalography and automated time tracking of computer use}
\def\mytitlesv{Klassificering av hjärnaktivitet med elektroencephalografi och automatiserad tidsspårning av datoranvändning}

\def\myauthor{Erik Bjäreholt}
\def\myorcid{0000-0003-1350-9677}
\def\myemail{erik@bjareho.lt}

\definecolor{LTHblue}{RGB}{0,0,128}
\definecolor{LTHbronze}{RGB}{156,97,20}

\newcommand\myshade{85}
\colorlet{mylinkcolor}{violet}
\colorlet{mycitecolor}{YellowOrange}
\colorlet{myurlcolor}{Aquamarine}

\hypersetup{%
  linkcolor=.,
  citecolor  = mycitecolor!\myshade!black,
  urlcolor = LTHblue,
  colorlinks = true,
}

% From: https://tex.stackexchange.com/a/173862/36302
\setbeamercolor*{structure}{bg=white,fg=LTHblue}

\setbeamercolor*{palette primary}{use=structure,fg=structure.fg,bg=structure.bg}
\setbeamercolor*{palette secondary}{use=structure,fg=white,bg=structure.fg!75}
\setbeamercolor*{palette tertiary}{use=structure,fg=white,bg=structure.fg!50!black}
\setbeamercolor*{palette quaternary}{fg=white,bg=black}

\setbeamercolor{section in toc}{fg=LTHblue,bg=white}
\setbeamercolor{alerted text}{use=structure,fg=structure.fg!50!black!80!black}

\setbeamercolor{titlelike}{fg=structure.fg!50!black}
\setbeamercolor{frametitle}{bg=white,fg=structure.fg}

\setbeamercolor*{titlelike}{parent=palette primary}

\bibliography{zotero}
\bibliography{misc}

% Add all used references to the cited category
\DeclareBibliographyCategory{cited}
\AtEveryCitekey{\addtocategory{cited}{\thefield{entrykey}}}

% Turn off split bibliography warning
\BiblatexSplitbibDefernumbersWarningOff

\defbibheading{notcited}{\section*{Further Reading}}


% Formatting
\setlength{\parindent}{0pt}
\setlength{\parskip}{1em}

%Information to be included in the title page:
\title{\mytitleen}
\subtitle{\ \\ Master's thesis presentation}
\author{\myauthor}
\institute{Department of Computer Science\\Faculty of Engineering\\Lund University}
\date{October 28, 2021}

% Set font size for captions
%\usepackage{caption}
%\captionsetup{format=plain, font=scriptsize,labelfont=scriptsize}
\setbeamerfont{caption}{size=\scriptsize}

\newif\ifplacelogo{}  % create a new conditional
\placelogotrue{}      % set it to true
\logo{\ifplacelogo\includegraphics[height=1.5cm]{LUlogoRGB.png}\fi}

\begin{document}

\frame{\titlepage}

\begin{frame}
    \begin{center}
        Abstract
    \end{center}
    {
        \scriptsizeWe investigate the ability of EEG to distinguish between different activities users engage in on their devices, building on previous research which showed a considerable difference in brain activity between code- and prose-comprehension, as well as differences during code- and prose-synthesis. We perform a replication study and improve upon past results using state-of-the-art machine learning classifiers based on Riemannian geometry.

Furthermore, we extend the scope of previous work by introducing the automated time tracking application ActivityWatch, to track the device activities that the user is engaging in. This lets us label EEG data with naturalistic device activity, which we then use to train classifiers to discern activities such as code writing vs prose writing, or work vs media consumption. Our results indicate that a consumer-grade EEG device can discern between different activities that a user performs at the computer. Among other results, we show that not only can code and prose \emph{comprehension} be distinguished, but also code and prose \emph{writing}.

% Our results show... 
% Our results can be used by software engineers and knowledge workers seeking to...
% Our results pave the way to better understand the minds of computer users in general, and software engineers, in particular, especially at work.
% Our results indicate that a consumer-grade EEG device can provide actionable insights into what a user is doing at the computer. More specifically, we show that\ldots"

% Our results confirm the findings of the original study, i.e., EEG data can be used to distinguish between code and prose-comprehension. Furthermore, the classifier based on Riemannian geometry outperforms the bandpass-features used by Fucci et al.~in terms of classification accuracy.

% Our results indicate that a consumer-grade EEG device can provide actionable insights into what a user is doing at the computer. More specifically, we show that we can train a classifier to discern between several types of device activity.

% In our naturalistic experiments, our results indicate that a consumer-grade EEG device can discern between different activities that a user performs at the computer. Among other results, we show that not only can code and prose \emph{comprehension} be distinguished, but also code and prose \emph{writing}.

A full replication package, including source code and a sample dataset, is available at \href{https://github.com/ErikBjare/thesis}{github.com/ErikBjare/thesis}

    }
\end{frame}

\begin{comment}
    % What lead up to the thesis
    \begin{frame}{Preface}
        \framesubtitle{How it all began}

        \begin{itemize}
            \item<1-> Pre-2013 --- Learning programming \& machine learning  % and Quantified Self
            \item<2-> 2013 --- Starting university
            \item<3-> 2015 --- ActivityWatch
            \item<4-> 2017 --- Finding the thesis project
            \item<5-> 2018 --- First experience using EEG
            \item<5-> 2019 --- Re-finding the thesis project
            \item<6-> 2020 --- Starting the thesis work
        \end{itemize}
    \end{frame}
\end{comment}

\begin{frame}{Outline}
    \begingroup
        \setlength{\parskip}{0em}
        \tableofcontents
    \endgroup
\end{frame}

\section{Introduction}
\begin{frame}{Introduction}
    Before we begin, we will present two technologies used:

    \begin{itemize}
        \item Functional brain imaging
        \item Automated time trackers
    \end{itemize}
\end{frame}

\subsection{Functional brain imaging}
\begin{frame}{Introduction}
    \framesubtitle{Functional brain imaging}

    Functional brain imaging is used to measure aspects of brain function.

    Examples:
    \begin{itemize}
        \item Electroencephalography (EEG)
        \item Magnetoencephalography (MEG)
        %\item Hemoencephalography (HEG)
        \item Functional Magnetic Resonance Imaging (fMRI)
        \item Functional Near-Infrared Spectroscopy (fNIRS)
    \end{itemize}
\end{frame}

\begin{frame}{Introduction}
    \framesubtitle{Functional brain imaging $>$ EEG}

    Developments in EEG the last $\sim$10 years:

    \begin{itemize}
        \item Cost reduction
        \item Consumer availability
    \end{itemize}

    Rough timeline:

    \begin{itemize}
        \item 2013: OpenBCI kickstarter
        \item 2016: InteraXon releases the Muse
        \item 2021: Neurosity releases the Crown
    \end{itemize}
\end{frame}

\begin{frame}{Introduction}
    \framesubtitle{Functional brain imaging $>$ EEG}

    Applications:

    \begin{itemize}
        \item Clinical (sleep, epilepsy)
        \item Brain-Computer Interfaces
        \item Neurolinguistics research
        \begin{itemize}
            \item Discerning code vs prose comprehension
                \begin{itemize}
                    \item Using MRI, by Floyd et al.~\cite{floyd_decoding_2017}
                    \item Using EEG \& various biosensors, by Fucci et al.~\cite{fucci_replication_2019}
                \end{itemize}
        \end{itemize}
        \item Biofeedback / meditation aid
        \item Quantified self (measuring mood, focus)
    \end{itemize}
\end{frame}

\subsection{Automated time trackers}
\begin{frame}{Introduction}
    \framesubtitle{Automated time trackers}

    We use an automated time tracker to track which device activity a user is engaging in.

    Examples:
    \begin{itemize}
        \item Screen Time (Apple)
        \item Digital Wellness (Android)
        \item RescueTime (commercial use)
        \item TimeAware (research use)
    \end{itemize}
\end{frame}


\begin{frame}{Introduction}
    \framesubtitle{Automated time trackers}

    Issues with existing solutions:
    \begin{itemize}
        \item Data detail \& temporal resolution
        \item Source availability / licensing
        \item Privacy concerns
    \end{itemize}
\end{frame}

\placelogofalse{}
\begin{frame}{Introduction}
    \framesubtitle{Automated time trackers}

    \begin{columns}
        \begin{column}{0.5\textwidth}
            \hspace{0.4em}
            \begingroup
                \scriptsize
                Our solution:
                \\
            \endgroup
            \vspace{0.2em}
            \hspace{0.4em}
            \begingroup
                \large
                ActivityWatch
            \endgroup
            \vspace{0.5em}

            \begin{itemize}
                \item{\emph{``The world's best free \& open-source automated time tracker''}}
                \item Started in 2015 by me
                \item My brother joined in 2016
                \item ${>}100{,}000$ downloads
                \item ${>}90$ contributors
                \item Available on Windows, macOS, Linux, and Android.
            \end{itemize}

            \vfill
        \end{column}
        \begin{column}{0.6\textwidth}
            \includegraphics[width=\textwidth]{img/screenshot-aw-activity.png}
        \end{column}
    \end{columns}
\end{frame}
\placelogotrue{}

\section{Theory}
\begin{frame}{Theory}
    Now a very brief introduction to underlying theory within electroencephalography and machine learning.
\end{frame}

\subsection{Electroencephalography}
\begin{frame}{Theory}
    \framesubtitle{Electroencephalography}

    Electroencephalography works by measuring tiny amounts of electrical potential (voltage) on the skull, which is caused by the activation of underlying neurons.

    Measurements are taken about 256 times per second, using one or more electrodes.
\end{frame}

\placelogofalse{}
\begin{frame}{Theory}
    \framesubtitle{Electroencephalography}
    \vspace*{-5mm}

    \hspace*{-10mm}
    \includegraphics[width=\paperwidth]{img/muselsl-signal.png}
\end{frame}
\placelogotrue{}

\begin{frame}{Theory}
    \framesubtitle{Electroencephalography}
    The 10--20 system is a standard for electrode placements.

    \begin{adjustbox}{width=0.8\textwidth,center}
        \includegraphics{img/1020system-extended-with-extra-info.png}
    \end{adjustbox}
\end{frame}

\begin{frame}{Theory}
    \framesubtitle{Electroencephalography}
    ERPs, of Event-Related Potentials, are stereotyped responses to a stimulus.

    \begin{adjustbox}{width=\textwidth,center,float=table}
        \begin{table}
    \centering
    \begin{tabular}{ll}
        \toprule
        ERP & Elicited by
        \\
        \midrule
        N170 & Processing of faces, familiar objects or words.
        \\
        N400 & Words and other meaningful stimuli.
        \\
        P300 & Decision making, oddball paradigm.
        \\
        P600 & Hearing or reading grammatical errors and other syntactic anomalies.
        \\
        \bottomrule
    \end{tabular}
    \caption{Common \emph{Event-Related Potentials} (ERPs)}\label{table:erps}
\end{table}

    \end{adjustbox}
\end{frame}

\placelogofalse{}
\begin{frame}{Theory}
    \framesubtitle{Electroencephalography}
    Example analysis of the N170 ERP:
    
    \begin{adjustbox}{width=\textwidth,center,float=figure}
        \includegraphics[width=\textwidth]{img/n170_viz.png}
    \end{adjustbox}
\end{frame}
\placelogotrue{}

\begin{frame}{Theory}
    \framesubtitle{Electroencephalography}
    The signal can be broken down into constituent frequencies. They can be roughly grouped into frequency bands, which are associated with certain brain states.

    {
        \small
        \begin{tabular}{lll}
    \toprule
    Frequency band & Frequencies & Brain states \\
    \midrule
    Gamma ($\gamma$) & >35 Hz & Concentration \\
    Beta ($\beta$) & 16--35 Hz & Active, external attention, relaxed \\
    Sigma ($\sigma$) & 12--16 Hz & Sleep spindles \\
    Alpha ($\alpha$) & 8--12 Hz & Very relaxed, passive attention \\
    Theta ($\theta$) & 4--8 Hz & Deeply relaxed \\
    Delta ($\delta$) & 0.5--4 Hz & Sleep \\
    \bottomrule
\end{tabular}

    }
\end{frame}

\subsection{Machine learning}
\begin{frame}{Theory}
    \framesubtitle{Machine learning}

    Machine learning can be used to classify EEG signals.

    Common approaches: 
    \begin{itemize}
        \item Riemannian methods
        \item Deep learning
        \item Common Spatial Pattern
        \item Bandpower-features
    \end{itemize}
\end{frame}

\begin{frame}{Theory}
    \framesubtitle{Machine learning $>$ Riemannian geometry}
    
    Riemannian methods in EEG utilizes the spatial information encoded in covariance matrices to estimate the similarity between two signals.

    In the simple \emph{Minimum Distance to Mean} (MDM) method, covariance matrices for each class are averaged in Riemannian space. For a new signal's matrix, the distance to each class is calculated, and whichever class distance is smaller becomes the predicted class.
\end{frame}

\begin{frame}{Theory}
    \framesubtitle{Machine learning $>$ Riemannian geometry}
    
    \small The Riemannian distance metric $\delta_G$ for two symmetric positive definite matrices $A$ and $B$ (such as covariance matrices) is~\cite{grafarend_metric_2003}:

        \[ \delta_G(A, B) = \sqrt{\sum_{i=1}^N \log^2 \lambda_i (A, B) } \]
\end{frame}

\begin{frame}{Theory}
    \framesubtitle{Machine learning $>$ Riemannian geometry}
    {
        \scriptsize
        \begin{figure}[h]
    \centering
    \includegraphics[width=0.6\textwidth]{img/riemannian-tangent-space.png}
    \caption{Schematic representation of the symmetric positive definite matrix manifold, the geometric mean $G$ of two points and the tangent space at $G$. The geometric mean of these points is the midpoint on the geodesic connecting $C_1$ and $C_2$, i.e.\ it minimizes the sum of the two squared distances. The map from the tangent space to the manifold is an exponential map. The inverse map is a logarithmic map.}\label{figure:tangent-space}
    \source{Congedo et al.~\cite{congedo_riemannian_2017}}
\end{figure}

    }
\end{frame}

\section{Method}
\begin{frame}{Method}
    We perform two different experiments:

    \begin{enumerate}
        \item Controlled code vs prose experiment
        \item Naturalistic device use
    \end{enumerate}

    \includegraphics[width=\textwidth]{img/method.png}
\end{frame}

\subsection{Devices}
\begin{frame}{Method}
    \framesubtitle{Devices}

    \begin{columns}
        \begin{column}{0.33\textwidth}
            \vspace{9mm}
            \centering
            \includegraphics[trim=0 -50 0 200,clip,width=\textwidth]{./img/Muse-S.jpg}
            \\ Muse S
        \end{column}

        \begin{column}{0.33\textwidth}
            \vspace{4mm}
            \centering
            \includegraphics[trim=200 150 200 100,clip,width=3cm]{./img/crown-1.png}
            \\ Neurosity Crown
        \end{column}

        \begin{column}{0.33\textwidth}
            \centering
            \includegraphics[width=3cm]{./img/openbci-cyton.jpg}
            \\ OpenBCI Cyton + Ultracortex
        \end{column}
    \end{columns}

    \vspace{-5mm}
    \begin{adjustbox}{width=\textwidth,center,float=table}
        \begin{table}[H]
    \centering
    \begin{tabular}{llcrr}
        \toprule
        Manufacturer
        & Device
        & Channels
        & Sampling rate
        & Comfort
        \\
        \midrule
        InteraXon
        & Muse S (2020)
        & 4
        & 256Hz
        & High
        \\
        Neurosity
        & Crown (2021)
        & 8
        & 256Hz
        & Medium
        \\
        OpenBCI
        & Cyton (2013) + Ultracortex
        & 8--16
        & 125--250Hz
        & Low
        \\
        \bottomrule
    \end{tabular}
    \caption{Devices used}\label{table:devices}
\end{table}

    \end{adjustbox}
\end{frame}

\subsection{Collection}
\begin{frame}{Method}
    \framesubtitle{Collection}

    Next up: Collecting data for our code vs prose experiments, followed by naturalistic use experiments.
\end{frame}

\begin{frame}{Method}
    \framesubtitle{Collection $>$ Code vs prose}
    {
        \small
        \begin{figure}[H]
    \centering
    \begin{tabular}{cc}
        \includegraphics[trim=25 160 0 0,clip,width=0.45\linewidth]{img/final-1-1.png}
        &
        \includegraphics[trim=20 0 20 0,clip,width=0.45\linewidth]{img/bugs_1.PNG}
        \\
        (a) Code comprehension
        &
        (b) Prose review
    \end{tabular}
    \caption{Sample of the tasks used as stimuli.}\label{fig:tasks}
\end{figure}

    }
\end{frame}

\begin{frame}{Method}
    \framesubtitle{Collection $>$ Naturalistic}
    A single subject (me) measured EEG while engaging in natural device use (both work and leisure). We define 4 categories of device activity.

    \begin{adjustbox}{width=0.7\textwidth,center,float=table}
        \includegraphics{img/naturalistic-dayclass-dist.png}
    \end{adjustbox}
\end{frame}

\subsection{Analysis}
\begin{frame}{Method}
    \framesubtitle{Analysis}

    We train two classifiers:
    \begin{itemize}
        \item Riemannian geometry
        \item Bandpower-features (benchmark)
    \end{itemize}

    General software libraries used:  scikit-learn, numpy, pandas.

    Domain-specific libraries used:  pyriemann, MNE, yasa.
\end{frame}

\begin{frame}[fragile]{Method}
    \framesubtitle{Analysis $>$ The classifiers}

    Our Riemannian classifier pipeline is constructed like this:
    
\begin{minted}[fontsize=\scriptsize]{python}
from sklearn.pipeline import make_pipeline
from sklearn.linear_model import LogisticRegression
from pyriemann.estimation import Covariances
from pyriemann.spatialfilters import CSP
from pyriemann.tangentspace import TangentSpace

clf = make_pipeline(
    Covariances(),
    CSP(4, log=False),
    TangentSpace(),
    LogisticRegression(),
)
\end{minted}
\end{frame}

\begin{frame}{Method}
    \framesubtitle{Analysis $>$ The classifiers}

    Our bandpower-based classifier computes the bandpower of each frequency band, and puts the values and their ratios in a feature vector. 

    We then use common ML methods for the actual learning and classification.
\end{frame}


\begin{frame}{Method}
    \framesubtitle{Analysis $>$ Windows and epochs}
    To train on and classify the EEG signal, we first need to label it and split into fixed-size windows.

    We divide the EEG-signal into epochs (according to their stimuli markers), and then split those down into 5s windows which we train on.

    We then also aggregate the predictions back into their epochs by taking the mean prediction of each window in the epoch, yielding predictions for entire epochs.
\end{frame}

\begin{frame}{Method}
    \framesubtitle{Analysis $>$ Performance evaluation}
    To evaluate our classifiers, we need a suitable performance metric and cross-validation method.
\end{frame}

\begin{frame}{Method}
    \framesubtitle{Analysis $>$ Performance evaluation}
    \begin{itemize}
        \item Studies using EEG often use \emph{balanced accuracy} (BAC).
        \item Balanced accuracy deals with imbalanced datasets.
    \end{itemize}

    \vspace{2em}

    For binary classification, BAC is defined as:
    \[ BAC = \frac{Sensitivity + Specificity}{2} = \frac{\frac{TP}{TP + FN} + \frac{TN}{TN + FP}}{2} \]
    This implies that, for the binary case, $BAC = 0.5$ is no better than chance.
\end{frame}

\begin{frame}{Method}
    \framesubtitle{Analysis $>$ Performance evaluation $>$ Validation}
    To ensure our classifier generalizes across subjects, we perform \emph{Leave-One-Run-Out} (LORO) cross validation.

    \begin{adjustbox}{width=0.6\textwidth, center, float=table}
        \definecolor{train}{RGB}{163, 206, 255}
\definecolor{test}{RGB}{255, 181, 84}
\begin{tabular}{lcccc}
    \toprule
           & Subject 1 & Subject 2 & Subject 3 & Subject 4 \\
    \midrule
    Fold 1 & \cellcolor{test}     & \multicolumn{3}{c}{\cellcolor{train}} \\
    Fold 2 & \cellcolor{train} & \cellcolor{test}     & \multicolumn{2}{c}{\cellcolor{train}     } \\
    Fold 3 & \multicolumn{2}{c}{\cellcolor{train}     } & \cellcolor{test}     & \cellcolor{train} \\
    Fold 4 & \multicolumn{3}{c}{\cellcolor{train}} & \cellcolor{test}     \\
    \bottomrule
\end{tabular}

        %\captionof{figure}{Example of Leave-One-Run-Out cross validation with 4 subjects. For each fold, subjects marked \textcolor{NavyBlue}{\textbf{blue}} are used for training and subjects marked \textcolor{BurntOrange}{\textbf{orange}} are used for testing.}
    \end{adjustbox}

    For our naturalistic device use experiment we instead use Stratified K-Fold cross-validation, as there is only one subject.
\end{frame}

\placelogofalse{}
\begin{frame}{Method}
    \framesubtitle{Comparison with previous studies}
    \vspace*{-10mm}
    \begin{adjustbox}{width=\textwidth,center,float=table}
        % Keep \begin{table} outside to let it work in both thesis and presentation
\rowcolors{4}{gray!25}{white}
\begin{tabular}{llll}
    \toprule
    \multirow{2}{*}{\textbf{Setting}} & \multicolumn{3}{c}{\textbf{Study}} \\
    \cmidrule(lr){2-4}
    & \makecell[c]{\textbf{This study}} & \makecell[c]{\textbf{Fucci et al.} (2019)} & \makecell[c]{\textbf{Floyd et al.} (2017)} \\
    \midrule
    Experiment site & Lund Univ. (Sweden) & Univ.\ of Bari (Italy) & Univ.\ of Virginia (USA)  \\
    \# Participants & 9 & 28 & 29 \\
    Participants experience & Grads & Undergrads & Grads \& Undergrads \\
    \# Tasks & Variable & 36 tasks & 27 tasks \\
    Task type & \makecell[l]{Code comprehension \\ Prose review} & \makecell[l]{Code comprehension \\ Prose comprehension} & \makecell[l]{Code comprehension \\ Code review \\ Prose review} \\
    Physiological signal & Neural & \Gape[0pt][2pt]{\makecell[l]{Neural \\ Skin \\ Heart}} & Neural \\
    Physiological measure & EEG & \makecell[l]{EEG \\ EDA \\ HR, HRV, BVP} & BOLD \\
    Device & Muse S & \Gape[0pt][2pt]{\makecell[l]{BrainLink Headset \\ Empatica wristband}} & fMRI \\
    Classifier & Riemannian geometry & 8 algorithms & Gaussian Process \\
    Classifier validation & LORO-CV & \Gape[0pt][2pt]{\makecell[l]{LORO-CV \\ Hold-out}} & LORO-CV \\
    Classifier metric & Balanced accuracy (BAC) & Balanced accuracy (BAC) & Balanced accuracy (BAC) \\
    \bottomrule
\end{tabular}
\rowcolors{2}{}{}
% Caption is kept outside due to issues with adjustbox as used in presentation
%\caption{Comparison of this study's method with previous studies.}\label{table:compare-method}

    \end{adjustbox}
\end{frame}
\placelogotrue{}

\section{Results}
\begin{frame}{Results}
    \begin{itemize}
        \item Controlled code vs prose experiment
        \item Naturalistic device use
    \end{itemize}
\end{frame}

\begin{frame}{Results}
    \framesubtitle{Controlled code vs prose experiment}
    Our results are:
    {\scriptsize
     \begin{table}[h]
    \centering
    \begin{tabular}{lcccc}
        \toprule
        & \multicolumn{2}{c}{\textbf{Riemannian}} & \multicolumn{2}{c}{\textbf{Bandpower}} \\
        \cmidrule(lr){2-3}
        \cmidrule(lr){4-5}
        \textbf{Subject} & Window-level & Epoch-level & Window-level & Epoch-level \\
        \midrule
        \#0 & 0.673 & 0.727 & 0.511 & 0.541 \\
        \#1 & 0.895 & 0.955 & 0.689 & 0.809 \\
        \#5 & 0.616 & 0.542 & 0.628 & 0.750 \\
        \#6 & 0.864 & 0.908 & 0.739 & 0.737 \\
        \#7 & 0.749 & 0.900 & 0.733 & 0.733 \\
        \midrule
        Median & 0.749 & 0.900 & 0.689 & 0.737 \\
        \bottomrule
    \end{tabular}
    \caption{The balanced accuracy for each LORO fold/subject. Excluding subjects 3, 4, 8, 9, and 10.}\label{table:bac-selective}
\end{table}

    }
\end{frame}

\begin{frame}{Results}
    \framesubtitle{Controlled code vs prose experiment}
    Compared to previous studies:
    {
        \scriptsize
        \begin{table}[h]
    \begin{center}
        \begin{tabular}{lcccc}
            \toprule
            & \multicolumn{2}{c}{\textbf{This study}} & \multirow{2}{*}{\textbf{Fucci et al.}} & \multirow{2}{*}{\textbf{Floyd et al.}} \\
            \cmidrule(lr){2-3}
            & Riemannian & Bandpower & & \\
            \midrule
            Overall & 0.75 & 0.69 & 0.66 & 0.79 \\
            \bottomrule
        \end{tabular}
        \caption{Result comparison between the previous studies and this study. Best balanced accuracy scores are reported. For this study, we used the best window-level score. For Fucci et al.\ we chose the best EEG-only score.}\label{table:compare-results}
    \end{center}
\end{table}

    }
\end{frame}

\begin{frame}{Results}
    \framesubtitle{Naturalistic device use}
    Our naturalistic device use results:

    \begin{adjustbox}{width=0.8\textwidth, float=table}
        \begin{tabular}{llrr}
    \toprule
    \textbf{Experiment} & \textbf{Score} & \textbf{Support} & \textbf{Hours} \\
    \midrule
    Programming vs Writing & 0.676 & (1386, 209) & 2.22h \\
    Programming vs Twitter & 0.695 &  (1386, 949) & 3.24h \\
    Programming vs YouTube & 0.672 &  (1386, 266) & 2.29h \\
    Twitter vs Writing & 0.833 &  (949, 209) & 1.61h \\
    Twitter vs YouTube & 0.604 & (949, 266) & 1.69h \\
    YouTube vs Writing & 0.889 &  (266, 209) & 0.66h \\
    \bottomrule
\end{tabular}

    \end{adjustbox}
\end{frame}

\section{Conclusions}
\begin{frame}{Conclusions}
    We conclude that we can discern code from prose using EEG, in both settings.

    We also find that\ldots
    \begin{itemize}
        \item Using a Riemannian approach outperforms the use of bandpower-features.
        \item It seems easier to discern work from leisure, than inter-work or inter-leisure tasks.
    \end{itemize}
\end{frame}

\subsection{Future work}
\begin{frame}{Conclusions}
    \framesubtitle{Future work}

    \begin{itemize}
        \item Collect more data
        \begin{itemize}
            \item NeuroTech Challenge
            \item More subjects for naturalistic use
        \end{itemize}
        \item Implement classification in moabb
        \item Create prose comprehension stimuli in English
        \item Use even better EEG devices, or even try fNIRS
        \item Turn brainwatch into a proper app to complement ActivityWatch
    \end{itemize}
\end{frame}

\section{Discussion}
\begin{frame}{Discussion}
    Time permitting, we will briefly go over threats and ethical considerations.
\end{frame}

%\subsection*{Applications}
%\begin{frame}{Discussion}
%    \framesubtitle{Applications}
%\end{frame}

\subsection*{Threats to validity}
\begin{frame}{Discussion}
    \framesubtitle{Threats to validity}

    \begin{itemize}
        \item Dataset
        \begin{itemize}
            \item Size
            \item Cherry-picked subjects
        \end{itemize}
        \item Stimuli
        \begin{itemize}
            \item Prose \emph{review} (not \emph{comprehension}) stimuli
        \end{itemize}
    \end{itemize}
\end{frame}

\subsection*{Ethical considerations}
\begin{frame}{Discussion}
    \framesubtitle{Ethical considerations}

    \begin{itemize}
        \item Research ethics
        \item Commercial ethics
    \end{itemize}
\end{frame}

\begin{frame}{Acknowledgements}
    \
    \\
    {\scriptsize \begin{itemize}
 \item My advisor Markus Borg~\orcid{0000-0001-7879-4371}.
 \item My brother Johan Bjäreholt and all the \href{https://activitywatch.net/contributors/}{ActivityWatch contributors}, for working with me all these years.
 \item The NeuroTechX crowd, specifically John Griffiths~\orcid{0000-0002-1764-2179} and Morgan Hough~\orcid{0000-0001-5256-413X}, for their support and time spent helping me.
 \item Pex Tufvesson and Carolina Bergeling at the Department for Automatic Control, for providing early guidance.
 \item Andrew Jay Keller at Neurosity, for gifting me a refurbished Notion DK1 to work with.
 \item Alex K. Chen, for referring me to all the right people.
 \item All the test subjects, for their time and interest.
 \item Everyone who has contributed to the open source tools I have used.
 \item Everyone who have supported me at LTH\@.
 \item Friends and family, for their neverending love and support.
\end{itemize}
}
\end{frame}

\placelogofalse{}
\begin{frame}{Appendix I}
    \framesubtitle{Data overview}
    {
        \tiny
        \begin{figure}
    \centering
    \includegraphics[width=\linewidth]{img/timebars.png}
    \caption{Visualization of the labeled data with classifications from one example subject-fold. Shows the \emph{Image} (stimuli), the class \emph{Label} for that stimuli (\textcolor{NavyBlue}{\textbf{blue}} is code, \textcolor{BurntOrange}{\textbf{orange}} is prose), the \emph{Predicted} class, whether the prediction is \emph{Correct}, the \emph{Subject}, the \emph{Split}/Fold (\textcolor{NavyBlue}{\textbf{blue}} shows the training set, \textcolor{Goldenrod}{\textbf{yellow}} the test set), and our threshold measure for signal \emph{Quality} (\textcolor{Green}{\textbf{green}} indicates acceptable quality). The x-axis is the window index, sorted by acquisition time.
    \\
    \vspace{0.5em}
    It can be seen that (1) subjects \#3 and \#4 have bad signal quality, and have therefore been excluded from the training set. (2) The subjects \#9 and \#10 have also been excluded from training due to issues during data collection. (3) For subject \#1 the stimuli images were not shuffled. (4) Subject \#0 appears twice, as they did two sessions (using unseen stimuli).}\label{fig:timebars}
\end{figure}

    }
\end{frame}
\placelogotrue{}

\section*{References}
\begin{frame}[allowframebreaks]{References}
    \AtNextBibliography{\scriptsize}
    \printbibliography[category=cited]
\end{frame}

\end{document}
