\chapter{Conclusions}

Our results successfully replicate previous research, showing that it is possible to distinguish between reading code and prose using EEG, and improves upon the state of the art in this regard by achieving a roughly $\sim$75\% balanced accuracy (using CV-LORO). However, the reader should note limitations of our study in Section~\ref{section:threats}.

Furthermore, our naturalistic experiments indicate it is possible to distinguish many other device activities from each other using consumer-grade EEG devices. Among these some data seems to suggest that EEG is sufficient to not only pick up differences in code vs prose \emph{comprehension} but also in \emph{writing}.

\section{Future work}

Future work could be to integrate the codeprose task into \texttt{moabb} to make it easier to replicate the results and evaluate new methods. As mentioned, the study would also benefit from more data collection, possibly through the NeuroTech Challenge Series (mentioned in Section~\ref{section:ntcs}).

As mentioned in the method, this study uses prose \emph{review} images from Floyd et al., as opposed to the prose \emph{comprehension} images (in Italian) used by Fucci et al. Ideally, one should create an english prose comprehension set of stimuli images, similar to the ones used by Fucci.
