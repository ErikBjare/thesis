We investigate the ability of EEG to distinguish between different activities users engage in on their devices, building on previous research which showed a considerable difference in brain activity between code- and prose-comprehension, as well as differences during code- and prose-synthesis. We perform a replication study and improve upon past results using state-of-the-art machine learning classifiers based on Riemannian geometry.

Furthermore, we extend the scope of previous work by introducing the automated time tracking application ActivityWatch, to track the device activities that the user is engaging in. This lets us label EEG data with naturalistic device activity, which we then use to train classifiers to discern activities such as code writing vs prose writing, or work vs media consumption. Our results indicate that a consumer-grade EEG device can discern between different activities that a user performs at the computer. Among other results, we show that not only can code and prose \emph{comprehension} be distinguished, but also code and prose \emph{writing}.

% Our results show... 
% Our results can be used by software engineers and knowledge workers seeking to...
% Our results pave the way to better understand the minds of computer users in general, and software engineers, in particular, especially at work.
% Our results indicate that a consumer-grade EEG device can provide actionable insights into what a user is doing at the computer. More specifically, we show that\ldots"

% Our results confirm the findings of the original study, i.e., EEG data can be used to distinguish between code and prose-comprehension. Furthermore, the classifier based on Riemannian geometry outperforms the bandpass-features used by Fucci et al.~in terms of classification accuracy.

% Our results indicate that a consumer-grade EEG device can provide actionable insights into what a user is doing at the computer. More specifically, we show that we can train a classifier to discern between several types of device activity.

% In our naturalistic experiments, our results indicate that a consumer-grade EEG device can discern between different activities that a user performs at the computer. Among other results, we show that not only can code and prose \emph{comprehension} be distinguished, but also code and prose \emph{writing}.

A full replication package, including source code and a sample dataset, is available at \href{https://github.com/ErikBjare/thesis}{github.com/ErikBjare/thesis}
