We investigate the ability of EEG to distinguish between different activities users engage in on their devices, building on previous research which showed a considerable difference in brain activity between code- and prose-comprehension, as well as differences during code- and prose-synthesis. We perform a replication study and improve upon past results using state-of-the-art machine learning classifiers based on Riemannian geometry.

Furthermore, we extend the scope of previous work by introducing the automated time tracking application ActivityWatch, to track the device activities that the user is engaging in. This lets us label EEG data with naturalistic device activity, which we then use to train classifiers to discern activities such as code writing vs prose writing, or work vs media consumption.

Code and a sample dataset is available at \href{https://github.com/ErikBjare/thesis}{github.com/ErikBjare/thesis}.
