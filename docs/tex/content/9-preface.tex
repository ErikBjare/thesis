\section*{Preface}

When I started university in 2013, I already knew that one day I wanted to work with brain-computer interfaces. During high school I had been soaked in transhumanist and hacker culture, spending more time online than in school.

I had come to the realization that the field was bound to fundamentally change how we interact both with computers and each other. However, I was aware that the field was in its earlier phases, and the work that needed to be done was outside of my expertise at the time.

So I asked how can I set myself up to be in a position to contribute to this impactful field of research. My answer at the time was to learn about data collection and analysis generally, and a couple years later applied my skills building ActivityWatch, as I figured the best approximation about what goes on in my head is what I'm looking at on my computer screen.

What I thought would be a small learning project soon took on a life of its own. People online started showing interest, and soon my brother joined in development. What started as a fun project to analyze my own device usage soon became a popular open source application with thousands of users. At some point it became apparent that once my Master's thesis was to be written, it would involve ActivityWatch one way or another.

As I was finishing up my final exams, I found an old paper in a moving box, it was a thesis proposal I had picked up at a career fair many years back. The proposal was from my advisor Markus Borg, and curious as I was, I contacted him and asked if it was still available. Markus enthusiastically replied, and helped me adapt his suggestion to combine it with my previous work on ActivityWatch.

As my thesis work progressed, I learned a lot about brain imaging and BCIs in general, and EEG in particular.

This thesis work has served as my first proper contribution to the field, all according to the plan I had made at the start of my university degree.

The Oxford English Dictionary defines `thesis' as ``a long essay or dissertation involving \emph{personal research}, written by a candidate for a university degree''. I can not think of more ``personal research'' than research in quantified self with personal data.
